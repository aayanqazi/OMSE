\documentclass[11pt]{article}

\usepackage{enumerate}
%\usepackage{todonotes}
\usepackage{times}

\makeatletter 
\def\s@btitle{\relax} 
\def\subtitle#1{\gdef\s@btitle{#1}} 
\def\@maketitle{% 
  \newpage 
  \null 
  \vskip 2em% 
  \begin{center}% 
  \let \footnote \thanks 
    {\LARGE \@title \par}% 
                \if\s@btitle\relax 
                \else\typeout{[subtitle]}% 
                        \vskip .5pc 
                        \begin{large}% 
                                \textsl{\s@btitle}% 
                                \par 
                        \end{large}% 
                \fi 
    \vskip 1.5em% 
    {\large 
      \lineskip .5em% 
      \begin{tabular}[t]{c}% 
        \@author 
      \end{tabular}\par}% 
    \vskip 1em% 
    {\large \@date}% 
  \end{center}% 
  \par 
  \vskip 1.5em} 
\makeatother 

% Remove section numbers
\setcounter{secnumdepth}{0}     % From headings
\def\thechapter{}               % From TOC

\begin{document}

\title{OMSE 510 - Distributed Teams \\ Best Practices Article Analysis}
\subtitle{Building Trust and Cooperation through Technology Adaptation in Virtual Teams: Empirical Field Evidence}
\author{Ben Straub}
\maketitle

This study was concerning the relationship between trust, leadership style, and technological
adaptation in distributed teams.  The two leadership styles contrasted are Theory X
(i.e. command-and-control, where decisions flow from the top of the org-chart) and Theory Y (the job
of the manager is to give people the resources they need and to get out of the way), and the
technology being adapted is the tools the team uses to communicate.

The study has three hypotheses.

\begin{itemize}
\item \textbf{H1:} Virtual team (VT) leader interventions employing a Theory X approach will lead to
  lower levels of trust than a Theory Y approach.
\item \textbf{H2:} VT leader interventions employing a Theory Y approach will lead to higher levels
  of technology adaptation than interventions employing more of a Theory X approach.
\item \textbf{H3:} Technology adaptation will be positively related to increased trust and
  cooperation in VTs.
\end{itemize}

The data was gathered between 2004 and 2005 using a retrospective-interview method, and includes
accounts of interventions by VT leaders.  Six judges then proceeded to classify and rate each
intervention for several criteria, including whether that intervention was Theory X or Y, as well as
several characteristics of the outcome of that intervention.  ``Trust'' was measured based on
high-level indicative behaviors of the group, and ``technology adaptation'' was indicated by
installation, removal, or customization of an information tool.

The first hypothesis was \textit{not} supported by the data.  It turns out that both X- and Y-style
leadership actions have almost exactly the same effect on trust in a VT.  

The second hypothesis \textit{was} supported.  There was a strong correlation between Y-style
leadership interventions and technology adaptation.

The third hypothesis \textit{was} supported.  There was a strong correlation between technology
adaptation and VT trust and cooperation.

This suggests that, while X-style leadership can indeed foster trust and cooperation within a team,
Y-style leadership can induce the same amount of trust while at the same time driving the effective
use of communication tools.  X seems to encourage direct human interactions; Y-style leaders let the
engineers choose the methods, and engineers almost always choose automated tools.

\section{Summary}

The article suggests that command-and-control leadership is just as effective at fostering trust in
a distributed team as facilitative leadership, but it doesn't tend to produce more effective use of
communication tools.  If virtual team communication is to be managed with automated tools, Theory
Y-style leadership should be used to manage it.


\end{document}
