\documentclass[11pt]{article}

\usepackage{enumerate}
\usepackage{todonotes}
\usepackage{amssymb}
\usepackage{times}
\usepackage{array}
\usepackage{longtable}
\usepackage{fullpage}

% Remove section numbers
\setcounter{secnumdepth}{0}     % From headings
\def\thechapter{}               % From TOC

\begin{document}



\title{OMSE 535 Project 4 \\ Unit Testing Plan and Test Cases}
\author{Ben Straub}
\maketitle

\section{Test Plan}

\textsc{Strategy:} Use black-box testing to ensure the consistency and correctness of the interface
as defined in \texttt{stack.h}.  White-box testing is used to exercise any code paths not covered by
black-box testing.

\begin{center}
  \begin{tabular}[h]{|l|l|}
    \hline
    \multicolumn{2}{|c|}{\bf Black-box Testing}                                                                                                    \\
    \hline \hline
    \textbf{Input}                                               & \textbf{Expected Result}                                                        \\
    \hline \hline
    None                                                         & \texttt{isempty()} returns non-zero.                                            \\
    \texttt{37}                                                  & \texttt{pop()} returns \texttt{37}                                              \\
    \texttt{10}, \texttt{20}, \texttt{30}                        & Calling \texttt{pop()} returns \texttt{30}, then \texttt{20}, then \texttt{10}. \\
    \texttt{5}                                                   & \texttt{isempty()} returns zero.                                                \\
    \texttt{push()} 1,000 times, then \texttt{pop()} 1,000 times & \texttt{isempty()} returns non-zero.                                            \\
    \hline \hline
    \multicolumn{2}{|c|}{\bf White-box Testing}                                                                                                    \\
    \hline \hline
    \textbf{Input}                                               & \textbf{Expected Result}                                                        \\
    \hline \hline
    \texttt{pop()}                                               & Program doesn't crash.  Return value is undefined.                              \\
    \texttt{while(1) \{ push(10); \}}                            & Program gracefully quits when memory is exhausted.                              \\
    \hline
  \end{tabular}
\end{center}

\section{Coverage}

Since the code includes no branches, a simple call to each function is sufficient to achieve 100\%
coverage, so the black-box test suite is more than enough for this purpose.  The (unchecked) return
codes from \texttt{malloc}, and the dereferencing of \texttt{NULL} are covered by the white-box
tests.

\section{Test report log}

\begin{center}
  \begin{tabular}[h]{|l|l|}
    \hline
    \multicolumn{2}{|c|}{\bf Black-box Testing}                                                                                                    \\
    \hline \hline
    \textbf{Input}                                               & \textbf{Result}                                                                 \\
    \hline \hline
    None                                                         & \texttt{isempty()} returns non-zero.                                            \\
    \texttt{37}                                                  & \texttt{pop()} returns \texttt{37}                                              \\
    \texttt{10}, \texttt{20}, \texttt{30}                        & Calling \texttt{pop()} returns \texttt{30}, then \texttt{20}, then \texttt{10}. \\
    \texttt{5}                                                   & \texttt{isempty()} returns zero.                                                \\
    \texttt{push()} 1,000 times, then \texttt{pop()} 1,000 times & \texttt{isempty()} returns non-zero.                                            \\
    \hline \hline
    \multicolumn{2}{|c|}{\bf White-box Testing}                                                                                                    \\
    \hline \hline
    \textbf{Input}                                               & \textbf{Result}                                                                 \\
    \hline \hline
    \texttt{pop()}                                               & {\color{red} Program crashes.}                                                  \\
    \texttt{while(1) \{ push(10); \}}                            & {\color{red} Program crashes.}                                                  \\
    \hline
  \end{tabular}
\end{center}

\end{document}