\documentclass[letterpaper,11pt]{article}

\usepackage[pdfborder=false]{hyperref} % Remove borders from PDF links
\usepackage[all]{hypcap}
\usepackage[utf8x]{inputenc}
\usepackage{ucs}
\usepackage{graphicx}
\usepackage[]{todonotes}

%\usepackage{times}
\usepackage{mathpazo}

\setlength{\parindent}{0pt}     % for less indentation
\setlength{\parskip}{1em}       % for less whitespace

\newcommand{\notdone}{\todo[color=red]{}}

% Remove numbering from subsections
\setcounter{secnumdepth}{1}

% Counters for commonalities and variabilities
\newcounter{type}
\newcounter{item}[type]
\newcommand{\com}[1]{\stepcounter{item}\item[\Alph{type}\arabic{item}.]{#1}}
\begin{document}

\author{Ben Straub\\OMSE 532}
\title{FWS Commonality Analysis}
\maketitle

\section{Introduction}


Floating Weather Station (FWS) buoys are deployed at sea and periodically
report the current wind speed via messages sent by radio. Each member of the
FWS family contains an on-board computer that controls the operation of the
buoy while it is at sea.
 
The purpose of this analysis is to provide the following capabilities for the
FWS family of buoys:
 
\begin{itemize}
\item A way to specify the configuration of a particular buoy. 
\item A way to generate, for a specified buoy configuration, the software that
  controls a buoy while it is at sea.
\end{itemize}

\section{Overview}

This commonality analysis is concerned with the following issues:
 
\begin{itemize}
\item What equipment configurations should be accommodated?
\item What computing platforms should be used on buoys?
\item What capabilities will be needed to make buoys sufficiently reliable to
  perform their missions?
\end{itemize}
 
\subsection{Interfaces To Other Domains}

Floating Weather Stations interact with systems that are equipped to receive
the signals transmitted by the on-board radio transmitter. Such systems may be
shipboard, ground-based, or satellite-based. The software for the FWS domain
interfaces with the sensors that the FWS uses to monitor wind speed, and with
the transmitter that the FWS uses to send messages. \autoref{fig:iface} shows
theses domains and gives a brief indication of the nature of the
interface. For example, the Sensor domain receives commands from the FWS
software and sends data to it.

\begin{figure}[h]
  \centering
  \includegraphics[width=4in]{FWSInterfacingDomains.pdf}
  \label{fig:iface}
  \caption{FWS Software and Interfacing Domains}
\end{figure}

\section{Dictionary of Terms}

\begin{description}
\item[Sensor period] The number of seconds between sensor readings.
\item[Report period] The number of seconds between message transmissions.
\item[Sample] A single reading from a single sensor.
\item[Snapshot] This is calculated from several sensor readings that are taken
  at the same time, using a weighted average algorithm.  A snapshot relates to
  a single measure.
\item[Report measure] A weighted average of several snapshots.
\item[Report] A collection of report measures, intended to provide information
  about several aspects of the buoy's environment.
\item[Weighted average] Given a set of value/weight pairs:
  \begin{displaymath}
    \lbrace (v_1,w_1), (v_2,w_2), \ldots, (v_N,w_N) \rbrace
  \end{displaymath}
  where $w_i \geq 0$ for all $i$ in $[1..N]$, their \textit{weighted average}
  is given by
  \begin{displaymath}
    {\displaystyle\sum_{i=1}^N v_i \times w_i}
    \over
    {\displaystyle\sum_{i=1}^N w_i}
    \nonumber
  \end{displaymath}
\item[Wind speed] The speed of the wind in knots: nautical miles per hour.
\item[Air Temperature] The temperature of the air surround the buoy: degrees
  Celsius ($^{\circ}$C).
\end{description}

\section{Commonalities}
\setcounter{type}{3} % 'C'

The following statements are basic assumptions about the FWS domain; they are
true of all FWS systems.

\subsection{Behavior}

\begin{description}
  \com{At fixed intervals, the FWS transmits a report containing an
    approximation of the current conditions at its location.}

  \com{The FWS uses a weighted average algorithm to produce a single snapshot
    from multiple sensor readings.}

  \com{The FWS uses a weighted average algorithm to produce a single report
    measure from several snapshots.}

  \com{Each FWS report contains data for all measures that FWS has sensors
    for.}

  \com{FWS samples all sensors with equal frequency.}
\end{description}

\subsection{Devices}

\begin{description}
  \com{FWS is powered by an on-board battery.}

  \com{FWS is equipped with one or more types of sensor, each of which
    can monitor a single measure.}

  \com{For each measure (e.g. wind speed, or air temperature), the FWS is
    equipped with one or more sensors that monitor it.}

  \com{For each measure, FWS interfaces with one or more classes of sensor,
    divided by resolution or capability.}
\end{description}


\section{Variabilities}
\setcounter{type}{22} % 'V'
\setcounter{item}{0}

The following statements describe how a FWS may vary. 
 
\subsection{Behavior}

\begin{description}
  \com{The weights used in the weighted average algorithm may vary depending
    on a variety of factors (e.g. measure, sensor resolution, sensor period,
    etc.)}

  \com{The number of sensors in each class varies.}
    
  \com{The sensor period varies.}

  \com{The report period varies.}

  \com{The number of snapshots that are averaged for a single snapshot
    varies.}
  
  \com{The number of samples used to produce a single report measure varies
    with the report period and the sensor period.}

  \com{The data size for each report measure varies.}

\end{description}


\subsection{Devices}

\begin{description}
  \com{FWS may be equipped with sensors for a varying number of measures.}

  \com{For each measure, FWS may have varying numbers of sensors.}

  \com{For each measure, FWS may have varying numbers of sensor classes.}

  \com{FWS supports a variety of vendors, families, and models of sensor for
    any given measure.  Each of these factors will affect the interface and
    amount of data received from the sensor.}

  \com{FWS may be equipped with on-board batteries of varying capabilities.}
    
\end{description}

\end{document}
