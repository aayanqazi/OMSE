
\documentclass[11pt,letterpaper]{article}
\usepackage[utf8x]{inputenc}
\usepackage{ucs}
\usepackage{times}
\usepackage{graphicx}
\usepackage[]{todonotes}

% Remove section numbers
\setcounter{secnumdepth}{0}     % From headings
\def\thechapter{}               % From TOC

\setlength{\parindent}{0pt}     % for less indentation
\setlength{\parskip}{1em}       % for less whitespace

\begin{document}

\author{Ben Straub\\OMSE 532}
\title{Spam Filter Excercise 1.2}
\maketitle


% ----------------------------------------------------------------------------
\section{Overview}

The architecture proposed below can be installed as a self-contained system in
an existing business environment.  It does not replace the existing SMTP or
POP3/IMAP server for the organization, but instead acts as a proxy for both
sending and receiving mail.  The user agent will use this product as both the
outgoing and incoming mail server.  

In graphical form, the proposed architecture for the system looks something
like this:

\setlength\fboxsep{0pt}
\setlength\fboxrule{0.5pt}
\begin{center}
  \fbox{
    \includegraphics[width=4in]{assignment_part_2_diagram.pdf}
  }
\end{center}


% ----------------------------------------------------------------------------
\section{Components}

\subsection{Data Store}
Other modules use this one to store and retrieve email messages.  This
module's primary secret is how messages are stored, and its primary
responsibility is storage and retrieval of messages.

\subsection{SMTP Proxy}
This module exposes an SMTP interface to the network, and proxies SMTP
requests to the actual server.  This module's primary secret is the actual
location of the SMTP server, and its primary responsibilities are to proxy
requests from the user agent to the external server with minimal modification,
and to store all outgoing messages in the data store.

\subsection{Mailbox Manager}
This module exposes POP3 and IMAP interfaces to the network, and allows user
agents to retrieve their messages.  Its primary secret is the details of the
POP3 and IMAP protocols.

\subsection{Credential Store}
Other modules use this one to store and retrieve credentials for logging into
other systems.  Its primary secret is how credentials are stored, and its
primary responsibility is to protect these credentials from unwanted access.

\subsection{Config/Admin}
This module exposes an HTTP interface to the internal network for the purposes
of administration and user configuration.  Its primary secret is the visual
interface, and the details of the HTTP protocol.

\subsection{Fetcher}
This module retrieves messages from the actual mailbox server, using POP3 or
IMAP.  Its primary secret is the actual location of the mailbox server.

\subsection{Detector}
This module is responsible for the activity of spam detection.  Spam detection
takes the form of filling in a message header field named
``\texttt{x-abc-spam}''.  Its primary secret is the number and name of all the
filtering algorithms.

\subsection{Filters}
Each filter contains the secret of its own algorithm and method for detecting
spam.  Filters are given access to other messages (using the Data Store) and
any external information they may need (using the Request Proxy and Credential
Store).

\subsection{Request Proxy}
This module is responsible for fulfilling requests for data from outside the
system, and providing cached results given an acceptable staleness level.  The
primary secret of this system is the method of caching.


% ----------------------------------------------------------------------------
\section{Interfaces}

\subsection{Network interfaces}
All network interfaces comply with an internet standard.  Examples are SMTP,
POP3, IMAP, HTTP, and UDP.

\subsection{Data Store}
The data store exposes a programmatic interface for storing and retrieving
email messages.  Individual messages may be identified by a key; sets of
messages can be identified by a date range, the content of a header, or a
full-text search.

\subsection{Credential Store}
This interface is oriented towards logging into external services.  Wherever
possible, passwords will be encrypted, and they will never in any case be
returned in plain-text (even if they are stored as such).

\subsection{Request Proxy}
This module's interface mainly consists of a single method:
\texttt{retrieve(max\_age, url)}.  The \texttt{max\_age} parameter is in hours,
and the URL is a resource addressible over a number of protocols.

\subsection{Filters}
Each filter exposes \texttt{ISpamFilter}, which mainly consists of the
\texttt{isSpam(message)} method, which returns a boolean.  The
\texttt{message} parameter is a structured representation of the full message,
including a way to retrieve the attachments (if any).


% ----------------------------------------------------------------------------
\section{Rationale}

It's expected that additional filters will be added later, and in fact this
will most likely be where the most effort will be spent after version 1
ships.  To meet this need, the number of concepts needed to write a filter are
minimal; the simple \texttt{IFilter} interface which must be implemented, and
the Data Store and Request Proxy interfaces for storing and retrieving any
data that may be needed.

For more detail on rationale, see ``Spam Filter Excercise 1.2'', delivered on
April 24, 2011.


\end{document}
