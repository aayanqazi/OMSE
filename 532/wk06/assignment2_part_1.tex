\documentclass[letterpaper,11pt]{article}

\usepackage[pdfborder=false]{hyperref} % Remove borders from PDF links
\usepackage[all]{hypcap}
\usepackage[utf8x]{inputenc}
\usepackage{ucs}
\usepackage{graphicx}
\usepackage[disable]{todonotes}

%\usepackage{times}
\usepackage{mathpazo}

\setlength{\parindent}{0pt}     % for less indentation
\setlength{\parskip}{1em}       % for less whitespace

\newcommand{\notdone}{\todo[color=red]{}}

\begin{document}

\author{Ben Straub\\OMSE 532}
\title{Spam Filter Excercise 2.1}
\maketitle

\setcounter{tocdepth}{1}
\tableofcontents
\clearpage


% ----------------------------------------------------------------------------
\section{Overview}

This is the architecture document for the ABC Spam Filter system.  Note that
this document is intended for several audiences.

\textbf{If you are a developer}, you will be most interested in
\autoref{sec:mod} (\autopageref{sec:mod}) and \autoref{sec:interop}
(\autopageref{sec:interop}).  These sections describe the modular structure of
the system and its externally-visible interfaces.

\textbf{If you are in marketing or project management}, you will be most
interested in section~\ref{sec:ttm} (page~\pageref{sec:ttm}).  This section
contains information on creating smaller versions of the system, such that the
feature set for a given release may be chosen to fit the schedule available.

\subsection{Structures in this document}

The following structures are used to document the system architecture.  Each
view of the system satisfies the needs of a particular architectural driver,
and is intended for a specific audience.

\begin{center}
  \begin{tabular}[H]{lll}
    \textbf{Driver}  & \textbf{Views} & \textbf{Primary Stakeholder} \\
    \hline \hline                                                    
    Modifiability    & Module, Layer  & Developers and maintainers   \\
    Interoperability & Module         & Developers and maintainers   \\
    Time to Market   & Layer          & Marketing                    \\
  \end{tabular}
\end{center}

\subsubsection{Module View}

The module view shows the decomposition of the system into logical modules.
This structure is used to view the system through the lens of information
hiding.  Each module in this view represents some piece of the system that is
protecting the other modules from knowing its secret.

Note that most nodes on this type of diagram represents a runtime or
design-time construct; only the leaf nodes correspond to a piece of code.  The
goal of this view is not to show all the components of the system; rather, it
is to provide a logical guide to where changes should be encapsulated.  Things
about the system that are likely to change should map to a single module.

\subsubsection{Layer View}
This layer shows a high-level \textit{uses} structure, where the elements are
logical groups of modules.  Each layer has an external interface that its
dependents use, rather than referencing the internal modules directly.

Edges on this graph are interpreted as an ``is allowed to use the interface
of'' relation; layer A \textit{uses} layer B if the correctness of A requires
the presence of a correct version of B.  This also implies that B's
correctness does not depend on any quality of A.


% ----------------------------------------------------------------------------
\section{Driver: Modifiability}
\label{sec:mod}

\subsection{Overview}
\begin{figure}[h]
  \centering
  \includegraphics[width=4in]{ArchModuleDecomp.pdf}
  \caption{Module Structure}
  \label{fig:mod1}
\end{figure}

\begin{figure}[h]
  \centering
  \includegraphics[width=3.5in]{ArchUses.pdf}
  \caption{Layer Structure}
  \label{fig:layer1}
\end{figure}

In order to support the primary design goal of modifying the set of spam
classification algorithms (see \autoref{app:mod}), the classifiers have been
isolated from the rest of the system (and from each other) through the use of
an abstract interface.  \autoref{fig:mod1} shows the modular decomposition of
the system.

\subsection{Element Catalog}
\subsubsection{Module Structure}
\label{sec:modules}
This section describes the elements of the module structure shown in
\autoref{fig:mod1}.
\begin{description}
\item[Storage/Messages] The primary secret of this module is the storage
  mechanism for email messages.  The system is expected to store all incoming
  and outgoing messages in a form that is suitable for analysis by
  classifiers.
\item[Storage/Settings] The primary secret of this module is the storage
  mechanism for configuration settings.  This includes such items as system
  configuration values, user-specific settings, configuration for classifiers,
  and so on.
\item[Storage/Credentials] The primary secret of this module is the storage
  mechanism for credentials.  This includes such items as per-user login
  information for external IMAP/POP3/SMTP email services, as well as any
  credentials that may be necessary for the classifiers.  Credentials are
  exposed in such a way as sensitive information (e.g. passwords) are exposed
  in a write-only fashion as much as possible.
\item[Communication/Protocols] The primary secret of this module's submodules
  are the details of a particular networking protocol.  
\item[Communication/Incoming Mail] The primary secret of this module is the
  details of providing appropriate access to messages to email user-agents.
\item[Communication/Outgoing Mail] The primary secret of this module is the
  details of proxying outgoing mail messages to an external SMTP server, and
  retaining a copy of the outgoing message in the message store.
\item[Configuration/UI] The primary secret of this module is the details of
  the interface used by system users to customize available classifier or
  system settings, or to view messages in the message store.
\item[Configuration/Install Config] The primary secret of this module is the
  details of bootstrapping the system from static configuration files.  This
  is separate from the Storage/Settings module because it contains details at
  a lower level, such as the location of the settings database.
\item[Updating/Downloading] The primary secret of this module is the details
  of checking for updates, and downloading the update package to the local
  machine.
\item[Updating/Installing] The primary secret of this module is the details of
  installing updates.  This includes both ``hot'' and ``cold'' updates;
  updating a running system, as well installing an update that requires a
  system restart.
\item[Classification/Classifiers] The primary secret of each classifier is the
  details of the algorithm used to identify email as \textit{spam} or
  \textit{ham}.
\item[Classification/Scheduler] The primary secret of this module is the
  details of when to trigger system events, such as the running of classifiers
  on a set of messages, or checking for system updates.
\end{description}

\subsubsection{Layer Structure}
This section describes the elements of the layer structure shown in
\autoref{fig:layer1}.
\begin{description}
\item[Scheduler] This block corresponds with the Classification/Scheduler
  module in \autoref{fig:mod1}.  It drives periodic system activities, and
  \textit{uses} all the blocks that have periodic activities to run.
\item[Incoming / Outgoing] This block corresponds with the incoming and
  outgoing mail modules in \autoref{fig:mod1}.  It \textit{uses} the following
  other blocks:
  \begin{itemize}
  \item Scheduler --- New messages can trigger classification, as well as
    other activities.
  \item Storage/Protocols/Platform --- Message arrival and retrieval requires
    the use of storage facilities and network protocols.
  \end{itemize}
\item[Classification] This block corresponds with the stack of classifiers in
  \autoref{fig:mod1}.  Each classifier is allowed to use the storage,
  protocols, and platform services to perform the task of classifying a given
  message as \textit{spam} or \textit{ham}.
\item[Updates] The update system (``Updating'' in \autoref{fig:mod1}) is
  allowed to use the platform and protocol services to perform the task of
  downloading and installing system updates.
\item[Storage/Protocols/Platform] This block aggregates a set of system
  services that are in common use by all the other blocks.  It corresponds to
  the ``Storage,'' ``Communication/Protocols,'' and
  ``Updating/Installing/Platform'' blocks in \autoref{fig:mod1}.
\end{description}


\subsection{Interface Definition}
\label{sec:interface}
% ------------------------------------------------------------
\subsubsection{\texttt{ISpamClassifier}}
\begin{description}
\item[Module] Classifiers
\item[Service] Determine message class

  Analyze the contents of a message and determine if it is \textit{spam} or
  \textit{ham}.
\item[Syntax] \texttt{classify(msg)}
\item[Parameters] \hfill
  \begin{description}
  \item[\texttt{msg}] Reference to an email message object.
  \end{description}
\item[Return Type] \texttt{EMessageClassification}

  This type indicates the class of message.  It is defined as an enumeration
  to allow more sophisticated classification in the future.
\end{description}

% ------------------------------------------------------------
\subsubsection{\texttt{IScheduler}}
\begin{description}
\item[Module] Scheduler
\item[Service] Schedule a repeating task

  The scheduler allows for both repeating and one-off tasks.
\item[Syntax] \texttt{schedule\_repeating(task, schedule)}
\item[Parameters] \hfill
  \begin{description}
  \item[\texttt{task}] An \texttt{Action} which is to be executed.
  \item[\texttt{schedule}] An instance of a \texttt{Schedule} object

    This object contains rich information about timing.  See the interface
    definition for the \texttt{Schedule} type for more information.
  \end{description}
\item[Return Type] \texttt{void}

  This method does not return a value, but failures are indicated by throwing
  an instance of the \texttt{SchedulerException} class.  See its definition
  for more information.
\end{description}

% ----------------------------------------------------------------------------
\section{Driver: Interoperability}
\label{sec:interop}

\subsection{Overview}
\begin{figure}[h]
  \centering
  \includegraphics[width=4in]{ArchModuleDecomp.pdf}
  \caption{Module Structure}
  \label{fig:mod2}
\end{figure}

The design goal of standards-based interoperability with other systems (see
\autoref{app:interop}) is satisfied by using standardized protocols for
external communication.

\subsection{Element Catalog}
The \texttt{Incoming Mail} module communicates with user-agents using POP3 or
IMAP.  This should allow operation with nearly every email client, both
current and future.

The \texttt{Outgoing Mail} module proxies an sent message to an external
server using SMTP, the protocol that all Internet mail servers support.
Again, this should provide correct operation with most every outgoing server
product in existence.

See the contents of \autoref{sec:modules} for more details on the elements of
\autoref{fig:mod2}.

\subsection{Interface Definition}
See the contents of \autoref{sec:interface} for details on the elements of
\autoref{fig:mod2}.



% ----------------------------------------------------------------------------
\section{Driver: Time to Market}
\label{sec:ttm}

\subsection{Overview}
\begin{figure}[h]
  \centering
  \includegraphics[width=4in]{ArchModuleDecomp.pdf}
  \caption{Module Structure}
  \label{fig:mod3}
\end{figure}

The goal of bringing a product to market in a timely fashion (see
\autoref{app:ttm}) is accomplished by defining useful subsets of the full
system that may be shipped as a complete product.

See the contents of \autoref{sec:modules} for more details on the elements of
\autoref{fig:mod3}.

\subsection{Element Catalog}
There are several chunks of this architecture that are not strictly necessary
to its basic functionality.

\subsubsection{Classifiers}
One obvious point of contraction is the set of message classifiers.  Since the
system is designed to be able to update this set at any time, it's a
relatively straightforward matter to ship a minimal set of algorithms for
version 1, and plan on deploying more on a tight update schedule.  This is
indicated in \autoref{fig:mod3} by the color-coded subsets of classifiers.

\subsubsection{Host Platforms}
Another way the system requirements can be shrunk is by supporting a minimal
set of host platforms in version 1.  This is indicated in \autoref{fig:mod3}
by the color-coded platform modules under ``Updating/Installing''.  More host
platforms may be added at a later time.

\subsubsection{Configuration Interface}
Requirements could be deferred by eliding any runtime configuration.  This is
indicated by the orange blocks in \autoref{fig:mod3}.

\subsubsection{Outgoing Mail Proxy}
The system could be reduced by leaving out the inclusion of an outgoing-mail
SMTP proxy.  This is indicated in \autoref{fig:mod3} by the dark blue blocks.


% ----------------------------------------------------------------------------
\clearpage
\appendix
\section{Architectural Drivers}

\subsection{Modify filtering algorithms}
\label{app:mod}
A key business goal is to be able to perform hot updates to customer’s spam
filter applications. This includes being able to modify the algorithms
implementing spam filtering criteria, add new ones, and remove old ones.
 
Following the scheme in the text, this property would be a type of
modifiability. More specifically, it is the ability to change, extend, or
contract the spam filtering algorithms at run time (binding time). This sort
of modification is a form of perfective maintenance.

\begin{description}
\item[Source] Spam filter developer/maintainer responsible for perfective
  maintenance of the filtering algorithms.
\item[Stimulus] Change the run-time the spam filtering criteria by adding,
  removing, or modifying one of the spam filtering algorithms.
\item[Artifact] This scenario applies to modifying the spam filtering criteria
  based on the existing system context (inputs, outputs, and platform). For
  example, adding a new filtering algorithm to a set of algorithms that
  operate on the client’s inbox. (I.e., this does not include more general
  modifications such as adding new sources of input or output, or changing the
  platform the system runs on.)
\item[Environment] Hot-patch: updates must be made to the customer’s system on
  site while the spam filter and mail client are running. If the spam filter
  software must shut down to complete the update then it must automatically
  re- start itself.
\item[Response] Maintainers identify a new threat or possible improvement to
  the spam filtering criteria. A component implementing the criterion is
  created or modified. The component is tested for quality and
  compatibility. The component is configured for and deployed using the
  system’s automated update capability.
\item[Response Measure] \hfill
  \begin{enumerate}
  \item Cost: modifications to the filtering algorithms are easy to make.  In
    particular, less than 10% of the development time or cost of
    modification configuring the new capability to run with the existing
    software (conversely, the bulk of time is spent on developing and testing
    the algorithm itself).
  \item Deployability: the new capability can be automatically deployed using a
    common update mechanism. Deployment does not adversely impact the customer’s
    email capabilities for a significant length of time (say, 3 minutes).
  \end{enumerate}
\end{description}
 
Assumptions: We may wish to further refine this scenario into several more
detailed ones if we identify different classes of updates with different
characteristics. Here we assume that an update can use 1) existing inputs and
outputs, 2) the existing deployment mechanism, and 3) the existing spam
filtering infrastructure (e.g., the other parts of the system that retrieve or
modify email). Thus, we expect that only one module will be added or changed
and that none of the other interfaces will need to change. We may identify
other types of updates for which one or more of these assumptions are invalid.

\subsection{Interoperability}
\label{app:interop}
A key business goal is that the Spam Filter work with the most common email
clients.
\begin{description} 
\item[Source] Anticipated customer base.
\item[Stimulus] Identification of a mail client that the spam filter must work
  with.
\item[Artifact] Spam filter installation and configuration components. Mail
  client interface components (retrieve,
\item[Environment] Spam Filter install-time. Install or run-time configuration
  for mail client interoperation. Run time interoperation with mail client.
\item[Response] Provide interfaces and configuration utilities necessary to 1)
  install the spam filter as a client plug-in, 2) allow any necessary
  configuration of options to work with the mail client and 4) interoperate
  with the mail client to fully perform the spam filtering functions at run
  time.
\item[Response Measure] The Spam Filter system installs and operates
  seamlessly with each target mail client. These clients must include Outlook,
  Outlook Express, Eudora, and Pegasus (or some other standard set)
\end{description}
 
This general scenario can be refined into a set of more concrete, detailed
scenarios – one for each mail client used by a significant portion of the
target market. It could also be decomposed into different scenarios covering
the different binding times for the interoperability. For example, we can have
one scenario for the install with Outlook, one for configuration, and one for
interoperation (and the same for each client where these might have different
results).
 
This scenario requires that the Spam Filter be tested with each client of
interest. This scenario does not assume that the Spam Filter must be
modifiable to accept new mail clients. This would be specified in a different
(Modifiability) scenario.

\subsection{Time to Market}
\label{app:ttm}
A key business goal is that the Spam Filter utility be deployed quickly to
establish market share and initiate a revenue stream. However, this business
goal and accompanying scenario must be addressed concurrently with the
following business goals and scenarios: 1) Rollout – ability to roll out new
versions of the system with additional features later 2) Extensibility – with
rollout, a capability necessary for the system’s upgrade path. In other words,
meeting time to market cannot sacrifice the capability to deploy future
versions of the system with enhanced capabilities.

\begin{description}
\item[Source] Marketing: analysis of market window for initial release
\item[Stimulus] Deadline: need to deploy initial set of capabilities within a
  market window.
\item[Artifact] Initial useful subset of the spam filter.
\item[Environment] Development time.
\item[Response] Define initial increment and set due date within market
  window.  Develop to schedule and requirements.
\item[Response Measure] \hfill
  \begin{enumerate}
  \item Spam filter meets functional and quality requirements for the initial
    release.
  \item Delivery on or before the market due date.
  \end{enumerate}
\end{description}
 
This scenario assumes that marketing identifies 1) the necessary requirements
(functional and quality) for an initial release and 2) that marketing defines
the market window by which the first increment must be delivered. This
scenario might be refined by identifying more than one subset and
corresponding market window.
 
It actually does not make a lot of sense to define this kind of quality
requirement as a “scenario” since one cannot execute it except in the general
course of developing the product. It would be more natural to simply describe
this in terms of the requirements for the initial increment and its due date.

\end{document}
