
\documentclass[11pt,letterpaper]{article}
\usepackage[utf8x]{inputenc}
\usepackage{ucs}
\usepackage{times}

% [disable] to turn in.
\usepackage[]{todonotes}

% Remove section numbers
\setcounter{secnumdepth}{0}     % From headings
\def\thechapter{}               % From TOC

\setlength{\parindent}{0pt}     % for less indentation
\setlength{\parskip}{1em}       % for less whitespace

\begin{document}

\author{Ben Straub\\OMSE 532}
\title{Spam Filter Excercise 1.1}
\maketitle


% ----------------------------------------------------------------------------
\section{Business Rationale}

In identifying the business opportunity for an ABC spam filter, several
high-level requirements have been determined.  Using these, in conjunction
with the data we have on existing ABC clients, we can reduce the set of
suitable architectures.  Below is a partial list of the business concerns
which have informed the decision-making process.

\begin{enumerate}
\item The solution should be a component independent of other ABC products.
  This allows it to be part of a prepackaged solution, bundled with other
  package upgrades to increase margin, or sold a la cart.
\item Given the diverse set of software being used by ABC's clients (both
  current and future), the solution should communicate using existing
  standards, and be compatible with as many standards-compliant servers and
  clients as possible.
\item The solution should take the form of a service running on a machine
  which is under the control of the customer.  This is the solution most
  compatible with the various regulatory and internal security requirements
  that our customers face.
\end{enumerate}


% ----------------------------------------------------------------------------
\section{Architectural Design Goals}

In order to fully meet the needs of the customer as well as allow future
versions of this product to be developed, the architecture will be developed
towards the following criteria.

\begin{enumerate}
\item \textbf{Standards Compliancy.}

  ABC's clients have a diverse ecosystem of email software, and it's essential
  that this product operate correctly in that environment.  In addition, ABC
  has no interest in entering the market of proprietary email products.  These
  two considerations mean that a solution should comply with widely-accepted
  standards and protocols.

\item \textbf{Adaptability.}

  Spam evolves, and this product must be capable of adapting as time goes on.
  This includes future product updates (which should happen on a short cycle),
  as well as customization of an installed system.

\item \textbf{Performance.}

  Email is a mission-critical service, and some of the customers requesting
  this product have very high email volumes.  The use of this system
  \textit{must not} adversely affect the day-to-day business of the company
  that purchased it, which means that per-message latency must be kept to a
  minimum.  

\end{enumerate}

% ----------------------------------------------------------------------------
\section{Components and Relations}

The representation of the architecture should contain the following
structures:

\begin{enumerate}
\item Modular decomposition
\item Concurrency
\item Client-server
\item Repository
\end{enumerate}

% ----------------------------------------------------------------------------
\section{Evaluation}

The system architecture is evaluated to fitness to the design goals using the
above structures.  Standards compliancy is evaluated by how the system will
communicate with external servers, as represented by the client-server
structure.  Adaptability is evaluated using the modular decomposition, and
modeling a new type of spam, or a refinement on old spam that isn't covered by
the current system.  Performance is evaluated using hte concurrency and
repository structures, making sure that network bottlenecks have been
accounted for, and that work is distributed to reduce latency.

\end{document}
