\documentclass[11pt]{article}

\usepackage[]{todonotes}
\usepackage{times}
\usepackage{fullpage}
\usepackage{tabulary}
\usepackage[pdfborder=false]{hyperref}

% Remove section numbers
\setcounter{secnumdepth}{0}     % From headings
\def\thechapter{}               % From TOC

\setlength{\parindent}{0pt}
\setlength{\parskip}{11pt plus 3pt minus 2pt}

\newif\ifblockcomment
\blockcommentfalse


\begin{document}

\title{Home Security System\\Function Point Estimation}
\author{Estimated by Ben Straub}
\date{July 26, 2009}
\maketitle


% Assignment: see ass2-setup.pdf


\section{Assumptions and Influences}
\todo[inline]{Base data and critical assumptions shaping the estimate (show the work done in
  calculating the function points, explain why program characteristics where low, medium or high
  complexity)}

\section{Component Estimates}

\subsection{Rationale}

\textbf{System Config} --

\textbf{Main Processor} --

\textbf{Panic Button Controller} --

\textbf{Door/Window Alarm} --

\textbf{Wireless Dial-out Controller} --

\textbf{Key Device Controller} --

\textbf{Motion Detector Controller} --

\textbf{Fire Detector Controller} --

\textbf{CO$_2$ Detector Conroller} --

\textbf{Light A/D Controller} -- 

\subsection{Ratings}

\todo[inline]{How to span columns? Abbreviations need a header.}
\begin{center}
  \begin{tabular}[h]{lllllll}
    {\bf Component}              & {\bf EI} & {\bf EO} & {\bf EQ} & {\bf IF} & {\bf EF} & {\bf Total} \\
    \hline \hline
    System Config                &          &          &          &          &          &             \\
    Main Processor               &          &          &          &          &          &             \\
    Panic Button Controller      &          &          &          &          &          &             \\
    Door/Window Alarm            &          &          &          &          &          &             \\
    Wireless Dial-out Controller &          &          &          &          &          &             \\
    Key Device Controller        &          &          &          &          &          &             \\
    Motion Detector Controller   &          &          &          &          &          &             \\
    Fire Detector Controller     &          &          &          &          &          &             \\
    CO$_2$ Detector Conroller    &          &          &          &          &          &             \\
    Light A/D Controller         &          &          &          &          &          & 
  \end{tabular}
\end{center}

\section{Calculations}

The formula for $AFP$ (Adjusted Function Points) in the SPR method is:

\begin{equation}
  AFP = UFP \times (0.4 + 0.1 \times (PC + DC))
\end{equation}

We choose a $PC$ (Program Complexity) of 2 and $DC$ (Data Complexity) of 3.

\begin{eqnarray}
  AFP & = & UFP \times (0.4 + 0.1 \times (2 + 3)) \\
      & = & UFP \times (0.4 + 0.1 \times 5) \\
      & = & UFP \times (0.4 + 0.5) \\
      & = & UFP \times 0.9 \\
\end{eqnarray}
\todo[inline]{Insert UFP here. Remove equation labels}

\end{document}

\section{Conversion to LOC}
Function points may be converted to an LOC estimate using the following calculation:

\begin{equation}
  
\end{equation}