\documentclass[11pt]{article}

\usepackage[pdfborder=false]{hyperref}
\usepackage[]{todonotes}
\usepackage{times}
%\usepackage{fullpage}
\usepackage{tabulary}
\usepackage{amsmath}

% Remove section numbers
\setcounter{secnumdepth}{0}     % From headings
\def\thechapter{}               % From TOC

\setlength{\parindent}{0pt}
\setlength{\parskip}{11pt plus 3pt minus 2pt}

\newif\ifblockcomment
\blockcommentfalse


\begin{document}

\title{Home Security System\\FP/LOC Estimation}
\author{Estimated by Ben Straub}
\date{July 26, 2009}
\maketitle


% Assignment: see ass2-setup.pdf


\section{Assumptions and Influences}
The assumption with the most impact on this estimate is that the logic necessary for the device
controllers is minimal.  These are modeled as binary triggers that are passed to the main processor
for handling.

Another assumption made is that there is an OS that provides both a filesystem and a Python runtime.
Omission of either of these represents a significant amount of complexity that will need to be moved
into the project code.

\section{Component Estimates -- Rationale}

\textbf{Main Processor} -- There are 5 low-complexity inputs from the various detector controllers,
as well as one medium-complexity input from the key controller.  Outputs include interfacing with
the dial-out device (medium), the key device (medium), and the light controller (low).  The
interaction with the system configuration can be modeled as a low-complexity query, and the module
does not interact with any files directly.

\textbf{System Config} -- This component has one medium-complexity output (the interface with the Main
Processor), and one medium-complexity internal file, the settings store.

\textbf{Controller Components} -- The internals of these are modeled as a low-complexity internal
file, and the output behavior as a low-complexity output.  The Key device has a higher external
output complexity rating.

\textbf{Light A/D Controller} -- This is modeled as a low-complexity input and a low-complexity
internal file.

\section{Component Estimates -- Ratings}

\begin{center}
  \begin{tabular}[h]{lcccccc}
                                 & \multicolumn{6}{c}{\bf UFP}                                        \\
    {\bf Component}              & {\bf EI} & {\bf EO} & {\bf EQ} & {\bf IF} & {\bf EF} & {\bf Total} \\
    \hline
    System Config                &          & 5        &          & 10       &          & 15          \\
    Main Processor               & 19       & 14       & 3        &          &          & 36          \\
    Panic Button Controller      &          & 4        &          & 4        &          & 8           \\
    Door/Window Alarm            &          & 4        &          & 4        &          & 8           \\
    Wireless Dial-out Controller &          & 4        &          & 4        &          & 8           \\
    Key Device Controller        &          & 5        &          & 4        &          & 9           \\
    Motion Detector Controller   &          & 4        &          & 4        &          & 8           \\
    Fire Detector Controller     &          & 4        &          & 4        &          & 8           \\
    CO$_2$ Detector Controller   &          & 4        &          & 4        &          & 8           \\
    Light A/D Controller         & 3        &          &          & 4        &          & 7           \\
    \hline
    {\bf Total}                  &          &          &          &          &          & {\bf 107}
  \end{tabular}
\end{center}

\section{Calculations}

The formula for $AFP$ (Adjusted Function Points) in the SPR method is:

\begin{align}
  AFP = UFP \times (0.4 + 0.1 \times (PC + DC)) \notag
\end{align}

We choose a $PC$ (Program Complexity) of 2, since there are no complicated algorithms in use, and
$DC$ (Data Complexity) of 3, which is nominal.

\begin{align}
  AFP & = UFP \times (0.4 + 0.1 \times (2 + 3)) \notag \\
      & = UFP \times (0.4 + 0.1 \times 5) \notag \\
      & = UFP \times (0.4 + 0.5) \notag \\
      & = UFP \times 0.9 \notag \\
      & = 107 \times 0.9 \notag \\
      & = 96 \notag
\end{align}

\section{System Size Estimate}
For this project, the team has chosen Python for implementation, given its high FP/LOC ratio, as
well as the ability to write extensions in C if greater control or performance are required.  Python
being an interpreted language, it's main weakness is performance, but this project is not
performance-constrained, and the hardware used for this product is capable of running a bare-bones
Linux system with a Python runtime.

Conversion factors from FP to LOC have been extrapolated for a variety of implementation languages.
Python is modeled for this estimate as equivalent to Perl, so the values are used for best-,
average-, and worst-case size estimates are 10, 20, and 30, respectively.  This results in the
following LOC estimate range, which are presented with two significant digits.

\begin{center}
  \begin{tabular}[h]{ccc}
    {\bf Best-Case} & {\bf Average-Case} & {\bf Worst-Case} \\
    \hline
    1000             & 1900               & 2900
  \end{tabular}
\end{center}

\end{document}
