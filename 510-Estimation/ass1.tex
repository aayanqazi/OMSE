\documentclass[11pt]{article}

\usepackage[disable]{todonotes}
\usepackage{times}
\usepackage{fullpage}
\usepackage{tabulary}
\usepackage[pdfborder=false]{hyperref}

\usepackage{tikz}
\usetikzlibrary{trees}


\setlength{\parindent}{0pt}
\setlength{\parskip}{11pt plus 3pt minus 2pt}

\newif\ifblockcomment
\blockcommentfalse


\begin{document}

\ifblockcomment
You are working with a development team on a large software project.  The project is creating a web
based three-dimensional modeling tool.  This system will allow registered users to create and
manipulate three-dimensional objects via a web browser.  Because this is a project with a great deal
of high-level (shareholder) interest, the team has been instructed to provide estimates for the
project at each stage of the software development life cycle.

Your job is not to help the team estimate the project but to help the team minimize the factors that
add to the uncertainty of their estimates at each stage of the project.  So for each stage of the
software development life cycle (you may assume a waterfall type life cycle), describe the factors
that you think will contribute to the uncertainty of the team’s estimate and why the factor impacts
the reliability of the estimate.  Then describe what the team can do to minimize the effect of the
given factors.  If there is no way to minimize the effect of a factor, explain why that is the case.

Please submit your recommendations to the team in a Word document via Blackboard’s assignment
mechanism.

Please contact Chris Gilmore (via email or Blackboard) for any questions you may have about the
assignment.
\fi

\title{OMSE 510 -- Software Estimation\\Assignment \#1}
\author{Ben Straub}
\date{July 12, 2009}
\maketitle


There are many sources of error in software estimation.  The purpose of this document is to provide
the project team with knowledge and recommendations for how to minimize the unnecessary
uncertainties for the estimation of this project.


\section{Planning Phase}

\begin{itemize}
\item {\bf Project planning} --- It's also imperative to have a complete project plan.  Have an
  experienced project manager produce or review the plan.
\item {\bf Missing activities} --- See McConnell page 45 (tables 4-3 and 4-4) for a list of
  development activities that are commonly overlooked when preparing a project plan.  Don't assume
  all team members will be working on the project 40 hours per week.
\item {\bf Personnel issues} --- Inexperienced or prima-donna team members can leach uncertainty into
  a project plan.  Minimize the number of rookie engineers on the project, and deal with personality
  problems early.
\item {\bf Tooling} --- This is the time to make sure your processes and tools are in place.
  Adequate source control is chief among these; choose a tool and workflow before any code is
  written.
\end{itemize}


\section{Requirements Phase}

\begin{itemize}
\item {\bf Unfinished requirements} --- It's imperative that requirements be of high quality for
  future consumption.  This means that they are investigated fully, with involvement from the
  customer.
\item {\bf Volatile requirements} --- If possible, produce a stable set (baseline) of requirements,
  and control all changes to that baseline with cost, effort, and schedule impact information.  It
  may be suitable to include requirements growth in the estimate.

  If the requirements are too volatile or the environment unsuitable for this practice, consider an
  alternate process model that better suits this situation.
\item {\bf Missing requirements} --- Check McConnell page 44 (table 4-2) for a list of requirements
  areas that are commonly overlooked.
  \end{itemize}


\section{Design and Implementation Phase}

\begin{itemize}
\item {\bf Abandoning planning} --- Don't stop planning activities because of schedule pressure.
  This will remove all of the estimation confidence you have.
\item {\bf Gold plating} --- Also pronounced YAGNI.  If it's not required for the current project,
  don't bother writing it.
\end{itemize}

\section{General Advice}

\begin{itemize}
\item {\bf Use the simplest method possible} --- Complex methods allow subjectivity to creep into
  estimates.  Use a simpler method if it will do.
\item {\bf Use multiple methods} --- If possible, compare the results from more than one estimation
  method.
\item {\bf Never give off-the-cuff estimates} --- A 15-minute estimate is much more accurate than a
  30-second estimate.
\item {\bf Don't negotiate estimates} --- They're the results of calculation, and should be treated
  as facts.  Instead, negotiate commitments and targets.
\item {\bf Use appropriate precision} --- If the uncertainty in an estimate is $\pm$2 months, don't
  give the mean in hours.
\item {\bf Know your math} --- Good estimation relies on the proper use of statistics, especially
  confidence levels and standard deviation.
\end{itemize}

\end{document}
