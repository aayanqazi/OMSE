\documentclass[11pt]{article}

\usepackage[]{todonotes}
\usepackage{times}
\usepackage{fullpage}
\usepackage[pdfborder=false]{hyperref}
\usepackage{tabularx}
\usepackage{tabulary}
\usepackage{amssymb}
\usepackage{ifpdf}

\usepackage{tikz}
\usetikzlibrary{trees}

% Enumeration environment: 1 -> 1.1 -> 1.1.1, etc
\makeatletter
\renewcommand{\labelenumii}{\arabic{enumi}.\arabic{enumii}}
\renewcommand{\labelenumiii}{\arabic{enumi}.\arabic{enumii}.\arabic{enumiii}}
\renewcommand{\theenumiii}{\arabic{enumiii}}
\renewcommand{\p@enumiii}{\arabic{enumi}.\arabic{enumii}.}
\makeatother

\newif\ifzero
\zerotrue

\setlength{\parindent}{0pt}
\setlength{\parskip}{11pt plus 3pt minus 2pt}

\begin{document}


%------------------------------------------------------------------------------- 

\title{Columbia AEHS\\Get-Well Plan}
\author{Ben Straub\\Initech Corporation}
\date{6/10/2009}
\maketitle
\thispagestyle{empty}

%-------------------------------------------------------------------------------


\vskip 1in

\todo{Revise}
The AEHS project has come to a crossroads, and if the project is to have any chance of succeeding,
some changes need to be made.  This document attempts to identify the reasons for the project's
current situation, and develops ideas for how to address the core issues.  A set of recommendations
is provided which will hopefully give the project a way to succeed.

\clearpage

%-------------------------------------------------------------------------------
\tableofcontents
\clearpage


%-------------------------------------------------------------------------------
\section{How we got here}

\todo[inline]{\ldots}

This project is in trouble.  Current estimates indicate that the specified work cannot be delivered
with less than a 6-month schedule slip, which cannot be prevented by simple measures such as
overtime, or extra manpower.  It's important to understand how this situation came to be if we are
to move forward in a constructive way.

One primary reason for this is a breakdown in communications.  The laundry that was aired at the
emergency meeting would not have been as dirty if JDI and CHG had been open, honest, and trusting
with each other.

Yet another reason is the lack of project-level flexibility.  From the outset, both the schedule and
budget were fixed, and though the initial requirements were vague, it has been made clear that a
scope reduction is not an option.  Realistically, AEHS is going to violate one of those constraints.
If we can realize this truth, we may get to choose which.  \todo{Revise}

%-------------------------------------------------------------------------------
\section{Where we go next}

\todo[inline]{Specific options/plans for progress.}

%-------------------------------------------------------------------------------
\section{Recommendations}

\todo[inline]{Tie it all together.}


% -------------------------------------------------------------------------------
\ifzero
\section{Week 10 summary}

\todo[inline]{Remove this section for submit (see ifzero).}

QN A: What do you believe are the key strategic reasons this project failed to be successful?  The
key strategic reasons why this project failed to succeed are as follows:

1. We think the primary reason for the project’s problems was the lack of flexibility. The
customer’s inflexibility with regard to scope and schedule was a big part in the strategy
failure. The project was overly constrained for the level of uncertainty in the requirements. A
project with a fixed budget and a fixed schedule, with poorly defined requirements is likely to be a
high risk project. The ability to prioritize requirements may have provided a majority of the
desired functionality on time. It turns out (from this case study) that schedule was not as
important as stated. Also, the budget was fixed from the start, and while JDI's wallet opened toward
the end of the project, it came too late in the project to help.

2. Another big reason is the requirements. Requirements uncertainty and volatility ultimately made
the team struggle. If not for technical reasons, emotionally as well. Poorly defined requirements
and the inability to achieve a requirements baseline early on in the project made it more
challenging. The customer’s requirements experts further exacerbated the problem and made it hard to
implement early requirements control and management.  The high level project objective was not
defined in detail prior to the beginning.

3. We also blame the failure on poor estimation techniques and inadequate tracking of progress
visibility. It also seems that communication channels between JDI and CHG weren't as open as they
could have been. If JDI were in constant communication about the status of the project, there would
never have been the need for a one-time urgent meeting. This is a big sign that people that were
making the decisions did not have visibility into the progress of the project. The last minute
“scramble” could have been avoided if they had seen it coming.

4. The project terms of reference didn’t put any responsibility on the customer to deliver the
needed collateral to JDI project team as a pre-requisite for a successful project. There was no
section of the document that talked about customer deliverables. Also, the definition of requirement
satisfaction (in the terms of reference) was very ambiguous and subjective as evidenced by the term
“fit to use”.

5. The project staffing and ramp up was very slow (too slow for a project of this scale and
constrained timeline). Unavailability of experienced staff to work on the project at the beginning
resulted in late staffing which further aggravated the situation

6. JDI’s inexperience in the healthcare domain and mobile device development posed additional
problems by requiring a prototyping stage and also meant that they really did not know what to
expect. The JDI startup team did not have the appropriate skill set to analyze the project needs and
to create a viable proposal. Additionally, the project team did not have all the skills that were
needed to develop such a system.

7. Due to all the uncertainties in the project, all the help the customer offered by the way of
additional programmers should have been accepted immediately and put to work and the customer should
have been asked for additional resources as needed. It seems like Columbia should be very familiar
with the HIPAA rules and regulations. If JDI had requested an expert from the Columbia then JDI
could have eliminated the evolutionary process components pertaining to healthcare.

QN B: What are the critical “get-well” actions you would have taken in January to maximize changes
of success?

The following critical “get-well” actions could have been taken in January to maximize the chances
of success:

1. Push harder for an extended schedule, for instance JDI could create a new project schedule with 3
cycles of delivery according to the requirements priority. Alternatively, JDI could settle for
delivering some of the less critical component as part of a future release if the customer pushes
back. JDI could implement the high priority requirements first so that even if the project ends up
being late the customer would be less impacted. This scheme should work well with iterative process
used for the project.

2. If the customer would not accept a schedule extension (which seems silly in retrospect, since
they were satisfied with the late project), then the SPM could present other alternatives such as a
scope restriction, or the phased deployment described above. This needs to be negotiated with the
customer so that their needs are met and everyone wins. It could be beneficial to propose trade-offs
whereby certain core capabilities, such as billing and patient records are not compromised whereas
other less critical areas could be delayed.

3. Consider outsourcing more. This project must have been broken down into modules and interfaces at
one point. JDI could outsource appropriate modules off to other companies more capable than
JDI. Many contracting companies specialize in doing “rescue” projects, so even some of the project
management could have been outsourced as needed. For instance, handheld devices should have been
outsourced freeing up much of the team that was tied up in this effort trying to come up to speed on
the technology.

4. The project team needs to identify components that can be replaced by COTs instead of developing
them in house. Scheduling, appointments, instant messaging, and other such components are potential
candidate for implementation via the application of COTS.

5. Send people to Columbia’s offices to observe work onsite and to proactively deduce requirements
to be confirmed with Columbia’s requirements experts to speed things up

6. Ask for more help in areas the customer had to address when creating their original system. JDI
should request a HIPAA expert from the customer. Columbia could work with JDI to more quickly get
the requirements completed without the spiral analysis. If this area is still having problems JDI
could switch from iterative development to an agile method. In this case Columbia could work more
closely with JDI and iteratively build the system they want leveraging their expertise in the health
care field.

7. Ask JDI senior management for 2 additional teams of programmers to work on other components of
the system that are fairly independent. Also, supplement the test team with more staff. The
developers on the test team can be used for defect fixing while the new staff can concentrate on
testing. The developers who are creating the system will not be as tied down with bug fixes.

8. Strive for a requirements base line and prioritization. This is the number one activity that the
SPM must drive, without a successful baseline and priorities, there is no chance that the project
will be able to get back on track. Once the requirements are baselined establish the change control
board (CCB) and also establish a risk management process to deal with all project risks on an
ongoing basis.

9. Automate as many processes as possible e.g. automated integration and building and automated
testing, bug tracking etc. Additionally, buy tools or make tools to help get the job done. It may
not have made much sense before to get small amounts of payoff with new tools, but now every savings
counts.

10. Keep up with progress reports and progress meetings and keep customer in the loop about the
status of the project at all times.  Also keep up with more active stakeholder management. Get both
the external and the internal stakeholders to support the SPM’s project rescue efforts.
\fi

\end{document}
