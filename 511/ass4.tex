\documentclass[12pt]{article}

\usepackage[disable]{todonotes}
\usepackage{times}
\usepackage[pdfborder=false]{hyperref}
\usepackage{enumerate}

% Remove section numbers
\setcounter{secnumdepth}{0}     % From headings
\def\thechapter{}               % From TOC

\makeatletter

% \iftodo command, binds to \usepackage[disable]{todonotes}
\newif\iftodo
\if@todonotes@disabled
\todofalse
\else
\todotrue
\fi

\makeatother

\setlength{\parindent}{0pt}
\setlength{\parskip}{11pt plus 3pt minus 2pt}

\begin{document}


%------------------------------------------------------------------------------- 

\title{Columbia AEHS\\Get-Well Plan}
\author{Ben Straub\\JDI Corporation}
\date{6/10/2009}
\maketitle
\thispagestyle{empty}

%-------------------------------------------------------------------------------


\vskip 1in

The AEHS project has come to a crossroads, and if the project is to have any chance of success,
changes must be made.  This document seeks to identify the reasons for the project's current
situation, develop ideas for how to address them, and recommend a course of action going forward.

\clearpage

%-------------------------------------------------------------------------------
\todo{Do we need/ want TOC?}
% \setcounter{tocdepth}{1}
% \tableofcontents
% \clearpage


%-------------------------------------------------------------------------------
\section{How we got here}

The AEHS project is in trouble.  Current estimates indicate that the specified work cannot be
delivered with less than a 6-month schedule slip, which cannot be prevented by simple measures such
as overtime, or extra manpower.  It's important to understand how this situation came to be if we
are to move forward in a constructive way.  This section is not intended as a finger-pointing
exercise, but rather as an objective look at the events that transpired.
\todo{}

{\bf Communications ---} The laundry that was aired at the emergency meeting would not have been as
dirty if JDI and CHG had been open, honest, and trusting with each other.  If project progress would
have been communicated in a more effective way, the meeting could have been avoided altogether.

{\bf Domain knowledge ---} This is JDI's first foray into the healthcare sector, and organizational
knowledge of that area is less than ideal.  Domain experts were not brought on until late in the
project, well past the time when their expertise would have been most beneficial.

{\bf Requirements ---} The lack of domain knowledge exacerbated the uncertainty in the requirements,
as well as the vagueness inherent in new platforms and advancing technology.  The initial contract
was a good starting point, but requirements work should have been performed alongside domain experts
from CHG, as well as some outside consultants.

{\bf Staffing --} The instability of the requirements led directly to an inability to effectively
staff the project -- the requirements simply weren't complete enough to have engineers start
designing and implementing to them.  An extended prototyping phase, which advanced the requirements
considerably, unfortunately absorbed too much development effort.

{\bf Flexibility ---} or lack thereof.  From the outset, both the schedule and budget were fixed,
and though the initial requirements were vague, it was made clear that a scope reduction is not an
option.  Realistically, AEHS is going to violate one of those constraints.  The only question left
here is which one we would prefer.

\clearpage


%-------------------------------------------------------------------------------
\section{We can do better}

None of the problems listed above are intractable, and reducing the impact of any of them would help
this project get back on track.  In this section, we explore some ways of getting past these
roadblocks.

\subsection{Communications}
Communication problems can be solved by providing more and/or better information.

To give \emph{more} information, we could open up the internal project-tracking dashboard and
defect-tracking database to CHG.  This has all the effects of taking a tour of a sausage factory,
but it may provide confidence that JDI is doing the job to the best of its ability, and removes lag
time between work being done and progress being reported to CHG.

\emph{Better} information takes two forms: information you can trust, and briefs of the large
amounts of data that are available.  Both of these could be produced by a CHG representative that
works part-time at JDI, whose responsibility is to act as a trusted source of data and reports.


\subsection{Domain Knowledge}
Applying domain knowledge takes one of two forms: using existing expertise, or acquiring more.

CHG has extensive knowledge of the healthcare domain, and their IT department could be a great help
to AEHS development.  One solution is to have JDI borrow several more engineers from CHG to work
full-time on AEHS, including at least one person with experience in implementing HIPAA-compliant
systems.

Another way to mitigate this problem is for JDI to acquire more domain knowledge.  This could be
achieved by sending some JDI engineers to CHG (to observe and interview end users), or to some
HIPAA-focused design and development training.


\subsection{Requirements}
Additional domain experts will go a long way towards tightening up the requirements, but it's
imperative to get final review and sign-off on the set of requirements that will be delivered.
Refining the requirements to the point where they can be estimated is necessary to baseline them,
which will help control the cost of changing them.  This should be a top priority.

Adjusting the project scope is another way to rein in the schedule.  This can be approached as
reducing the total scope of AEHS, or as a phased rollout of the full system.  Either option means
CHG must settle for less functionality in the short term, but this may be acceptable, depending on
priorities.  It will also mean deciding which requirements are more important than the others.

Parts of the requirements may be satisfied by acquiring COTS components.  Research and cost analysis
will be required to determine which components are available, how they would fit into the system as
it stands, and the amount of modification that will be necessary.

\subsection{Staffing}
The cross-staffing changes mentioned above will help with the staffing, but there is still too much
work for the current staff level.  One proposal is to bring on more development personnel from
inside JDI, which will either start some of the work that can be done in parallel with the existing
teams, or take over ownership and fix defects for subsystems that are mostly complete.  Some care
will need to be taken to moderate Brooks' law; the goal here should be effective application of
personnel, rather than reducing the schedule with brute force.

Outsourcing is an option considered by many organizations in this kind of situation, but
realistically, its application is limited.  Offloading development of the core system to another
organization would likely take longer than having JDI finish it; contract negotiations and rework
would neutralize any potential schedule gain.

One area where outsourcing shows some promise is in extended maintenance or support.  While JDI is
perfectly capable of performing these tasks, it is entirely up to CHG whether this avenue is to be
pursued.

\subsection{Flexibility}
In order for the project to have any chance of success, the project management team needs some
dimension of flexibility to deal with unforeseen circumstances.  CHG could provide this by answering
the following questions:

\begin{itemize}
\item Would you rather have the project on schedule at the cost of some functionality?  (The answer
  will likely depend on the functionality in question, so this is really about ten questions.)
\item Which would you choose: a project that is six months late but on budget, or on-time but with a
  50\% cost overrun?
\item Which requirements areas would you cut if you could shave 10\% from the budget?
\end{itemize}

This information is indispensable for producing any kind of get-well plan.  Knowing which of the
million things that go into a system are the most important (and which can be set aside) is
essential for delivering to a satisfied customer.


%-------------------------------------------------------------------------------
\clearpage
\section{Recommendations}
What follows are the recommendations of JDI, in priority order.  These are the actions which we feel
will have the most traction in getting the project under control, with the least possible cost and
risk.

\begin{enumerate}[\bf 1.]
\item Decide whether to pursue AEHS with reduced scope, perform a phased rollout of the full system,
  or extend the schedule.
  \begin{enumerate}[\bf a.]
  \item If with reduced scope, decide which subset is to be retained.
  \item If as a phased rollout, prioritize which subsets of the full requirements will be included
    with each phase, and when each phase will be delivered.
  \end{enumerate}
\item Get schedule/budget/scope prioritization matrix from CHG. 
\item Add two full development teams to the project, staffed mostly with JDI engineers.  Include at
  least one domain expert from CHG on each team.
\end{enumerate}

We feel that these actions stand the best chance of producing a satisfactory result for both JDI and
CHG.  While this project has been a challenge for both organizations, we hope this story will end
well for all concerned.



% -------------------------------------------------------------------------------
\iftodo
\section{Week 10 summary}

QN A: What do you believe are the key strategic reasons this project failed to be successful?  The
key strategic reasons why this project failed to succeed are as follows:

1.  We think the primary reason for the project’s problems was the lack of flexibility.  The
customer’s inflexibility with regard to scope and schedule was a big part in the strategy failure.
The project was overly constrained for the level of uncertainty in the requirements.  A project with
a fixed budget and a fixed schedule, with poorly defined requirements is likely to be a high risk
project.  The ability to prioritize requirements may have provided a majority of the desired
functionality on time.  It turns out (from this case study) that schedule was not as important as
stated.  Also, the budget was fixed from the start, and while JDI's wallet opened toward the end of
the project, it came too late in the project to help.

2.  Another big reason is the requirements.  Requirements uncertainty and volatility ultimately made
the team struggle.  If not for technical reasons, emotionally as well.  Poorly defined requirements
and the inability to achieve a requirements baseline early on in the project made it more
challenging.  The customer’s requirements experts further exacerbated the problem and made it hard
to implement early requirements control and management.  The high level project objective was not
defined in detail prior to the beginning.

3.  We also blame the failure on poor estimation techniques and inadequate tracking of progress
visibility.  It also seems that communication channels between JDI and CHG weren't as open as they
could have been.  If JDI were in constant communication about the status of the project, there would
never have been the need for a one-time urgent meeting.  This is a big sign that people that were
making the decisions did not have visibility into the progress of the project.  The last minute
“scramble” could have been avoided if they had seen it coming.

4.  The project terms of reference didn’t put any responsibility on the customer to deliver the
needed collateral to JDI project team as a pre-requisite for a successful project.  There was no
section of the document that talked about customer deliverables.  Also, the definition of
requirement satisfaction (in the terms of reference) was very ambiguous and subjective as evidenced
by the term “fit to use”.

5.  The project staffing and ramp up was very slow (too slow for a project of this scale and
constrained timeline).  Unavailability of experienced staff to work on the project at the beginning
resulted in late staffing which further aggravated the situation

6.  JDI’s inexperience in the healthcare domain and mobile device development posed additional
problems by requiring a prototyping stage and also meant that they really did not know what to
expect.  The JDI startup team did not have the appropriate skill set to analyze the project needs
and to create a viable proposal.  Additionally, the project team did not have all the skills that
were needed to develop such a system.

7.  Due to all the uncertainties in the project, all the help the customer offered by the way of
additional programmers should have been accepted immediately and put to work and the customer should
have been asked for additional resources as needed.  It seems like Columbia should be very familiar
with the HIPAA rules and regulations.  If JDI had requested an expert from the Columbia then JDI
could have eliminated the evolutionary process components pertaining to healthcare.

QN B: What are the critical “get-well” actions you would have taken in January to maximize changes
of success?

The following critical “get-well” actions could have been taken in January to maximize the chances
of success:

1.  Push harder for an extended schedule, for instance JDI could create a new project schedule with
3 cycles of delivery according to the requirements priority.  Alternatively, JDI could settle for
delivering some of the less critical component as part of a future release if the customer pushes
back.  JDI could implement the high priority requirements first so that even if the project ends up
being late the customer would be less impacted.  This scheme should work well with iterative process
used for the project.

2.  If the customer would not accept a schedule extension (which seems silly in retrospect, since
they were satisfied with the late project), then the SPM could present other alternatives such as a
scope restriction, or the phased deployment described above.  This needs to be negotiated with the
customer so that their needs are met and everyone wins.  It could be beneficial to propose
trade-offs whereby certain core capabilities, such as billing and patient records are not
compromised whereas other less critical areas could be delayed.

3.  Consider outsourcing more.  This project must have been broken down into modules and interfaces
at one point.  JDI could outsource appropriate modules off to other companies more capable than JDI.
Many contracting companies specialize in doing “rescue” projects, so even some of the project
management could have been outsourced as needed.  For instance, handheld devices should have been
outsourced freeing up much of the team that was tied up in this effort trying to come up to speed on
the technology.

4.  The project team needs to identify components that can be replaced by COTs instead of developing
them in house.  Scheduling, appointments, instant messaging, and other such components are potential
candidate for implementation via the application of COTS.

5.  Send people to Columbia’s offices to observe work onsite and to proactively deduce requirements
to be confirmed with Columbia’s requirements experts to speed things up

6.  Ask for more help in areas the customer had to address when creating their original system.  JDI
should request a HIPAA expert from the customer.  Columbia could work with JDI to more quickly get
the requirements completed without the spiral analysis.  If this area is still having problems JDI
could switch from iterative development to an agile method.  In this case Columbia could work more
closely with JDI and iteratively build the system they want leveraging their expertise in the health
care field.

7.  Ask JDI senior management for 2 additional teams of programmers to work on other components of
the system that are fairly independent.  Also, supplement the test team with more staff.  The
developers on the test team can be used for defect fixing while the new staff can concentrate on
testing.  The developers who are creating the system will not be as tied down with bug fixes.

8.  Strive for a requirements base line and prioritization.  This is the number one activity that
the SPM must drive, without a successful baseline and priorities, there is no chance that the
project will be able to get back on track.  Once the requirements are baselined establish the change
control board (CCB) and also establish a risk management process to deal with all project risks on
an ongoing basis.

9.  Automate as many processes as possible e.g.  automated integration and building and automated
testing, bug tracking etc.  Additionally, buy tools or make tools to help get the job done.  It may
not have made much sense before to get small amounts of payoff with new tools, but now every savings
counts.

10.  Keep up with progress reports and progress meetings and keep customer in the loop about the
status of the project at all times.  Also keep up with more active stakeholder management.  Get both
the external and the internal stakeholders to support the SPM’s project rescue efforts.
\fi

\end{document}
