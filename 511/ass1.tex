\documentclass[11pt]{article}

\usepackage[]{todonotes}
\usepackage{times}
\usepackage{fullpage}
\usepackage{tabulary}
\usepackage[pdfborder=false]{hyperref}

\begin{document}


%------------------------------------------------------------------------------- 

\title{Columbia AEHS\\Software Project Management Plan}
\author{Ben Straub\\Version 1.0}
\date{4/19/2009}
\maketitle
\thispagestyle{empty}

\vskip 1in
\todo[inline]{
  A short statement of the context and aim of this document should go here.
}

\vskip 1in
{\Large \bf Revision History}

\begin{center}
  \begin{tabulary}{\textwidth}{c|l|l|l}
    \bf{Version} & \bf{Description}                                  & \bf{Responsible Party} & \bf{Date} \\
    \hline \hline
    1.0          & Initial version of SPMP for comment by senior mgt & Ben Straub             & 4/19/2009
  \end{tabulary}
\end{center}

\vskip 1in
{\Large \bf Approval History}

\begin{center}
  \begin{tabulary}{\textwidth}{c|l|l|l}
    \bf{Version} & \bf{Description} & \bf{Responsible Party} & \bf{Date} \\
    \hline \hline
                 &                  &                        & 
  \end{tabulary}
\end{center}
\clearpage

%-------------------------------------------------------------------------------
\tableofcontents
\clearpage


%-------------------------------------------------------------------------------
\section{Overview and Project Summary}
\todo[inline]{This section of the plan should provide an overview of the entire project. Provide a brief introductory statement about the project here.
The Advanced E-Healthcare System (AEHS) is a major overhaul of the automated systems in use by the Columbia Healthcare Group. }

\subsection{Purpose, Context, Scope of Requirements, and Objectives}
\todo[inline]{Summarize the primary purpose, scope and objectives of the project.  When describing scope, express it by summarizing the requirements and how they may relate to the overall project context such as existing systems and operations.  When addressing context, identify key project stakeholders (customer, users, other affected people or companies), nature of the problem being addressed, and general approach that will be use to address the problem.}


\subsection{Assumptions, Constraints and Dependencies}
\todo[inline]{What are your known and/or assumed project priorities, constraints and dependencies?  Briefly address factors such as schedule, budget, requirements, tools, infrastructure and human resources available as well as systems, interfaces, software, and technology to be used.  Use a flexibility matrix to illustrate how you will balance the various project priorities.  }

\subsection{Project Deliverables}
\todo[inline]{Identify internal and external (customer) project deliverables including dates and/or project milestones.  Include internal plans and reports as well as customer deliverables such as documents, scripts, source code, executables on CD, database tables. Note that some of these documents may be provided to the customer and your own management as well as the team.  }


\subsection{Summary of Schedule and Budget}
\todo[inline]{Provide a top level description of the schedule and budget with major activities and milestones – leave the details to the appropriate sections below.}


\subsection{Evolution of this SPMP}
\todo[inline]{You should say something about how you plan to evolve the SPMP.  When do you plan to revisit it and update it?  How will you review and approve this document as you progress through project phases? }


\section{References}
\todo[inline]{You should also include a list of all key documents, web and other resources used and referenced in the project.  Don’t forget to keep this up-to-date with each significant release of the SPMP. }



\section{Definitions}
\todo[inline]{This should be a short list of unique project-specific key definitions and
  acronyms. Don’t include common terms like SPMP, SRS, SQA, SCM here (assume that all readers will
  know these). Remember that good practice is to define commonly used acronyms in (parentheses) the
  first time you use them (the title of this document illustrates this good practice).  }



\section{Project Organization}
\subsection{Organizational Interfaces (Internal and External)}
\todo[inline]{Show how you plan to organize your team to carry out the activities and tasks identified above.  You should include an organization chart showing your team member’s roles and responsibilities and how they relate to each other.  You can also show “interfaces” to your customer, users (if applicable), your “senior management”, and others supporting your project.  Identify and describe internal organizational units/teams (e.g. requirements, design, development, integration, testing, SQA, SCM, etc.) as well as external organizations and essential contact information (e.g. customers, users, contractors, vendors etc.) }


\subsection{Project Roles and Responsibilities}
\todo[inline]{Describe the roles and responsibilities of all teams and team members, if applicable, including those of external participants.}



\section{Managerial Process}
\textit{TBD}
% \subsection{Work Breakdown and Tasks}
% Your SPMP should clearly show the assignment of team members to activities and tasks.  To do this, you need to identify all the activities and tasks for the project, in other words, develop a “work breakdown structure”.  You should also describe what the typical work package will contain (provide one representative example).  
% A straight forward approach is to list the activities and tasks you have already identified and add to these general effort tasks like project planning, project tracking, documentation, development environment support, and quality assurance.  
% \subsection{Dependencies}
% Show critical dependencies that will affect the ordering of tasks.
% \subsection{Resources}
% You should construct a table that identifies all of the key activities and tasks listing the team roles supporting each.  
% \subsection{Budget and Resource Allocation}
% Show the individual team member(s) allocated to each activity / task.  You should estimate the number of person days in the last column and find the total estimated resources to complete all project tasks.  You can use a nominal loaded salary to estimate the cost of all activities and tasks and produce a total labor cost for the project.  
% \subsection{Schedule}
% Create a schedule, preferably in the form of a Gantt chart.  This should show a bar for every activity and task identified above over the project time line.  Annotate the chart with all major milestones as well as weekly progress reviews.  
% \subsection{Monitoring, Reporting, and Control Mechanisms}
% Describe what progress and control metrics you will use to report progress across your teams, to senior management and to the customer. What will be included in your reports? How often will you deliver them and brief various stakeholders? 
% \subsection{Requirements Management and Control}
% One key risk is that the scope of the work may be larger than originally anticipated.  Briefly describe your approach for monitoring possible “scope creep” and negotiating required changes with the customer and your senior management.  
% \subsection{Risk Management}
% For each project risk, what do you plan to do to mitigate the risk? For example, what if your customer does not deliver certain documents or tools?  What if your sponsor drops out or does not provide as much guidance as you were hoping?  What if a key member of your team drops out, how will you cover this loss?  What if after some initial work you discover that there is more work than you thought?  How will you deal with this but still achieve a reasonable outcome?  What can you do to prepare for such eventualities? 




\section{Technical Processes}
\subsection{Process Model}
\todo[inline]{What is the process model you plan to follow (Waterfall, Incremental Delivery, Evolutionary, XP, Spiral)?  You should clearly identify your main phases, activities and milestones.  You should clearly show what you expect to achieve and deliver by each milestone, and by what date (a milestone every 2 weeks is reasonable).  At a minimum, your activities should include those for establishing the customer’s requirements, setting up your development environment, your architecture and design activities, coding, integration, and testing, and achieving customer acceptance.  }


\subsection{Methods, Tools and Techniques}
\todo[inline]{What processes and methods do you plan to use to specify your architecture / design, develop your code, create your builds, test your modules and subsystems, and deploy your deliverable software?  }
\subsection{Development Environment and Infrastructure}
\todo[inline]{Describe your development environment identifying the tools, languages, operating systems, and utilities (e.g. bug tracking) that will support your methods and tools above.}
\subsection{Acceptance Process}
\todo[inline]{Describe the process you will use to accept deliverables including software and documentation.  }



\section{Support Processes}
\textit{TBD}
% \todo[inline]{Describe support processes for the following:}
% \subsection{Software Configuration Management}n
% What will you do to ensure that you control versions of your documents and code?
% \subsection{Software Quality Assurance}n
% How will you ensure that your documents and code have been independently reviewed by someone on your team?
% \subsection{Documentation Management}n
% Describe how you will manage the key software documents produced by the project, in particular the SPMP, SRS, SAD, user documentation and other key customer deliverables.  How will you review, approve, and release these documents (what, who, how, when)?
% \subsection{Subcontractor Management}n
% Describe the work that will be contracted out, if any, deliverables, schedules for delivery, and subcontractor control mechanisms to be used to ensure timeliness and quality of deliverables. 




\section{Additional Supporting Plans}
\textit{TBD}
\todo[inline]{If required, additional plans can be inserted here.}




\section{Annexes}
\textit{TBD}
\todo[inline]{These should include details that would clutter the main sections.  All items herein should be references from somewhere in the main body of the document.  }



\end{document}