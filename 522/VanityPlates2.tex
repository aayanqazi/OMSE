\documentclass[11pt]{article}

\usepackage{bs-zed}

\begin{document}

\title{Vanity Plates II}
\author{Ben Straub}
\maketitle

\iftypechecker
Consider the situation from our earlier assignment on Oregon Vanity Plates. Again imagine that you
have been charged with implementing the GUI and input checking associated with step 1 of the
attached application form for an Oregon Vanity Plate.

This time, you decide to write a Z specification for your implementation as a first step. As before,
choose any input structure you want. Please do the specification in state-schema style. The output
should be a sequence of CHAR representing a valid plate to be printed. For invalid plates, you may
either output an error or simply not output anything until the plate is valid, depending on whether
your specification is interactive or not.

The two obvious styles of specification are to accept individual typed input characters in the style
of the text editor example from the course text, or to accept as input an already-edited sequence of
chars to validate.

Please discuss as desired on the course discussion list: this is not a competitive assignment.

I would guess that this project will take 2-8 hours, depending on how comfortable you are with Z at
this point. But that's pretty much a wild guess.

(see plateapp.pdf; due 2/24)
\fi

\section{English}
The instructions on the form can be interpreted as a set of high-level requirements.  Assuming that
the input system resolves issues of handwriting quality, it boils down to the following:

\begin{enumerate}
\item \label{e:len} There are at most six ``regular'' characters, which may include a space or
  full-hyphen.
\item \label{e:chr} If all six regular characters are used, an additional character may be
  inserted.  The only valid characters for this rule are the space and half-hyphen.
\item \label{e:pat} Plates of the forms \textit{\#\#\#-AAA} and \textit{AAA-\#\#\#} are prohibited,
  where \textit{A} is any letter and \textit{\#} is any number.
\end{enumerate}

\section{Z Types}

Before we can talk about valid plate strings, we need to talk about what kind of characters are
valid.  

\begin{zed}
  CHAR  ::= space | hyphen | halfhyphen | blank |           \\
  \t3 letter \ldata 1..26 \rdata | digit \ldata 0..9 \rdata \\
  DIGIT == \ran digit                                       \\
  LETTER == \ran letter                                     \\
  PUNCTALPHA == \{ space, hyphen \}                         \\
  PUNCTDEMI == \{ space, halfhyphen \}                      \\
  PUNCT == PUNCTALPHA \cup PUNCTDEMI                        \\
  STR == \seq CHAR
\end{zed}

The $CHAR$ type represents all valid characters for a plate.  We divide them into subsets to
facilitate rule validation later.  Note the presence of the $blank$ character, which is used for
recentering the printable characters, but does not participate in validation.

$PUNCTALPHA$ and $PUNCTDEMI$ define the punctuation characters that are valid in various conditions,
and $PUNCT$ defines the entire set of punctuation.  The type we'll use for input is a sequence of
elements of type $CHAR$.

If I were getting paid for this, I'd probably give helpful error messages; \textit{this} validator
has only two possible status values.

\begin{zed}
  STATUS ::= ok | error
\end{zed}



\section{Z Schemas}

In order to handle punctuation, we define functions to separate a $STR$ into $CHAR$s and
$PUNCT$s.

\begin{axdef}
  punct : STR \fun STR
  \where
  \forall str : STR; p : PUNCT; c : CHAR @                                        \\
  \t1 punct~\langle \rangle = \langle \rangle \land                               \\
  \t1 punct~(\langle p \rangle \cat str) = \langle p \rangle \cat punct~str \land \\
  \t1 punct~(\langle c \rangle \cat str) = punct~str
\end{axdef}

\begin{axdef}
  nopunct : STR \fun STR
  \where
  \forall str : STR; p : PUNCT; c : CHAR  @                    \\
  \t1 nopunct~\langle \rangle = \langle \rangle \land          \\
  \t1 nopunct~(\langle p \rangle \cat str) = nopunct~str \land \\
  \t1 nopunct~(\langle c \rangle \cat str) = \langle c \rangle \cat nopunct~str
\end{axdef}

Now we can define our validator's program state; the input is a string, and the output is a status
code.

\begin{schema}{ValidatorState}
  str? : STR \\
  status! : STATUS
\end{schema}

Next we define the catch-all error state, which we'll fall through into if none of our valid cases
apply.

\begin{schema}{ValidatorError}
  ValidatorState
  \where
  status! = error
\end{schema}

Now for the Rule \#\ref{e:len} validation: total length is at most 7, with at most one punctuation
character.

\begin{schema}{LengthIsValid}
  ValidatorState
  \where
  \#(nopunct~str?) \leq 6 \\
  \#(punct~str?) \leq 1   \\
  status! = ok
\end{schema}

According to rule \#\ref{e:chr}, the kinds of punctuation that are allowed depend on the length of
the string.  Full strings of six characters can only have one character from the $PUNCTDEMI$ set,
while shorter strings can have a single punctuation characters from the $PUNCTALPHA$ set.

\begin{schema}{FullStringPunctuationIsValid}
  ValidatorState
  \where
  \#(nopunct~str?) = 6 \\
  \#(punct~str?) = 1 \implies (punct~str?)(1) \in PUNCTDEMI
\end{schema}

\begin{schema}{SparseStringPunctuationIsValid}
  ValidatorState
  \where
  \#(nopunct~str?) < 6  \\
  \#(punct~str?) \leq 1 \\
  \#(punct~str?) = 1 \implies (punct~str?)(1) \in PUNCTALPHA
\end{schema}

Our last rule to validate is \#\ref{e:pat}, which is fairly straightforward; we simply assert that
the input string doesn't match the invalid patterns.

\begin{schema}{PatternIsValid}
  ValidatorState
  \where
  \forall a,b,c : LETTER; i,j,k : DIGIT @           \\
  \t1 (str? \neq \langle a,b,c,i,j,k \rangle) \land \\
  \t1 (str? \neq \langle i,j,k,a,b,c \rangle)
\end{schema}

\section{The End}

All that's left is to define the system in total.  Either all the validation rules pass, or the
output status is an error.

\begin{zed}
  Validator ==                                                  \\
  \t1(LengthIsValid \land FullStringPunctuationIsValid \land    \\
  \t2 SparseStringPunctuationIsValid \land PatternIsValid) \lor \\
  \t1 ValidatorError
\end{zed}

\end{document}
