\documentclass[11pt]{article}

\usepackage{bs-zed}

\begin{document}

Consider the situation from our earlier assignment on Oregon Vanity Plates. Again imagine that you
have been charged with implementing the GUI and input checking associated with step 1 of the
attached application form for an Oregon Vanity Plate.

This time, you decide to write a Z specification for your implementation as a first step. As before,
choose any input structure you want. Please do the specification in state-schema style. The output
should be a sequence of CHAR representing a valid plate to be printed. For invalid plates, you may
either output an error or simply not output anything until the plate is valid, depending on whether
your specification is interactive or not.

The two obvious styles of specification are to accept individual typed input characters in the style
of the text editor example from the course text, or to accept as input an already-edited sequence of
chars to validate.

Please discuss as desired on the course discussion list: this is not a competitive assignment.

I would guess that this project will take 2-8 hours, depending on how comfortable you are with Z at
this point. But that's pretty much a wild guess.

(see plateapp.pdf)

\end{document}
