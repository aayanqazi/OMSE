% Problems 6.5.1, 7.8.1, 7.8.2, and 7.8.4 from The Way of Z.

% Please do the problems as best as you can. 6.5.1 is pretty open-ended; we're not looking for some
% picayune detail, but for an important missing component of the specification. The use of pencil
% and paper plus a scanner is encouraged for 7.8.1 and 7.8.2, but you may also use LaTeX or any
% standard ASCII Z notation.

\documentclass[11pt]{article}
\usepackage{zed-csp}
\usepackage{times}
\usepackage{fullpage}
\begin{document}




\section{Problem 6.5.1}
\begin{quote} \textit{
    Something very important was left out of our state machine model. What was it? (It can be
    expressed in terms of the states and events we already defined.)
  } \end{quote}

There's no initial state. (Aside: this is given in Appendix F of the book.)

\section{Problem 7.8.1}

\begin{quote}
  \textit{Define the total operations T\_Backward and T\_Delete.  You may use any functions or
    operators defined in Appendix D. }
\end{quote}

Following the pattern for $T\_Forward$, we must first define the left arrow key:

\begin{axdef}
  left\_arrow : CHAR
  \where
  left\_arrow \notin printing
\end{axdef}

Thus left\_arrow is a character, but not the kind that prints.  Now for some schemas that we can
combine later.

\begin{schema}{BOF}
  Editor
  \where
  left = \langle \rangle
\end{schema}

This defines the ``beginning-of-file'' condition, where the $left$ list is empty.

\begin{schema}{LeftArrow}
  ch? : CHAR
  \where
  ch? = left\_arrow
\end{schema}

$LeftArrow$ is fulfilled when an input character is the $\leftarrow$ key.  

\begin{schema}{Backward}
  \Delta Editor \\
  ch? : CHAR
  \where
  ch? = left\_arrow \\
  left' = front(left) \\
  right' = \langle last(left) \rangle \cat right
\end{schema}

Backward means that we move the last character in $left$ to $right$.  Now we're ready to define
$T\_Backward$:

\begin{zed}
  T\_Backward \defs Backward \lor (BOF \land LeftArrow \land \Xi Editor)
\end{zed}

Here is the fast-forward for $T\_Delete$:

\begin{axdef}{}
  delete : CHAR
  \where
  delete \notin printing
\end{axdef}

\begin{schema}{}
  
\end{schema}

\section{Problem 7.8.2}

\begin{quote}
  \textit{Define the operation that handles input characters that are not handled by Insert,
    Forward, Backward, and Delete.}
\end{quote}


\section{Problem 7.8.4}

\begin{quote}
  \textit{How many distinct states are described by the Editor schema? The EOF schema? Assume that
    CHAR is implemented by the ASCII character set.}
\end{quote}



\end{document}
