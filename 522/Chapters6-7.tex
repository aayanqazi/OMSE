% Problems 6.5.1, 7.8.1, 7.8.2, and 7.8.4 from The Way of Z.

% Please do the problems as best as you can. 6.5.1 is pretty open-ended; we're not looking for some
% picayune detail, but for an important missing component of the specification. The use of pencil
% and paper plus a scanner is encouraged for 7.8.1 and 7.8.2, but you may also use LaTeX or any
% standard ASCII Z notation.

\documentclass[11pt]{article}
\usepackage{bs-zed}

\begin{document}

\title{Chapters 6 \& 7 Problems}
\author{Ben Straub}
\maketitle

\section{Problem 6.5.1}
\begin{quote} \textit{
    Something very important was left out of our state machine model. What was it? (It can be
    expressed in terms of the states and events we already defined.)
  } \end{quote}

There's no initial state. (Aside: this is given in Appendix F of the book.)

\section{Problem 7.8.1}

\begin{quote}
  \textit{Define the total operations T\_Backward and T\_Delete.  You may use any functions or
    operators defined in Appendix D. }
\end{quote}

\subsection{$T\_Backward$}

I'll save some space and re-use all of the definitions given in Chapter 7.

Following the pattern for $T\_Forward$, we must first define the left arrow key; it is a
non-printing character.

\begin{axdef}
  left\_arrow : CHAR
  \where
  left\_arrow \notin printing
\end{axdef}

Next we need a similar concept to EOF, so we won't crash if the user hits the left-arrow key at the
beginning of the file.

\begin{schema}{BOF}
  Editor
  \where
  left = \langle \rangle
\end{schema}

Now we build up the schemas we'll combine later. First, the operation when the user enters the
$\leftarrow$ key.

\begin{schema}{LeftArrow}
  ch? : CHAR
  \where
  ch? = left\_arrow
\end{schema}

Next is what happens when we want to move backward through the buffer.  We simply move the last
character from $left$ to the beginning of $right$.

\begin{schema}{Backward}
  \Delta Editor \\
  ch? : CHAR
  \where
  ch? = left\_arrow \\
  left' = front(left) \\
  right' = \langle last(left) \rangle \cat right
\end{schema}

Now we're ready to define $T\_Backward$ as the combination of the last three schemas:

\[ T\_Backward \defs Backward \lor (BOF \land LeftArrow \land \Xi Editor) \]

\subsection{$T\_Delete$}

Let's start with the axiomatic definition for the delete character, which is yet another
non-printing character.

\begin{axdef}
  del : CHAR
  \where
  del \notin printing
\end{axdef}

Now we can say what happens when it is entered; we remove the last character from $left$.

\begin{schema}{Delete}
  \Delta Editor \\
  ch? : CHAR
  \where
  ch? = del \\
  left' = front(left) \\
  right' = right
\end{schema}

Unless, of course, the cursor is at the start of the file.  But we've already defined $BOF$, so it
looks like we have everything we need.

\[ T\_Delete \defs Delete \lor (BOF \land Del \land \Xi Editor) \]

\section{Problem 7.8.2}

\begin{quote}
  \textit{Define the operation that handles input characters that are not handled by Insert,
    Forward, Backward, and Delete.}
\end{quote}

We'll define this operation to do nothing; if a character isn't $Insert$able, and isn't $del$ or the
left- or right-arrow keys, we'll ignore it.  

\begin{schema}{Ignore}
  Editor \\
  ch? : CHAR
  \where
  ch? \notin printing
\end{schema}

From here, it's straightforward to define the total version of our editor.  It's no Emacs, but
that's probably for the best---it's only week 3.

\[ T\_Editor \defs T\_Insert \lor T\_Delete \lor T\_Backward \lor T\_Forward \lor Ignore \]


\section{Problem 7.8.4}

\begin{quote}
  \textit{How many distinct states are described by the Editor schema? The EOF schema? Assume that
    CHAR is implemented by the ASCII character set.}
\end{quote}

\subsection{Editor}
Editor allows for any text of length up to $maxsize$, which is bounded at $2^{16}$.  $CHAR$ is
implemented as ASCII, which provides 94 printable characters\footnotemark.  In the largest case, the
number of states in $Editor$ is given by:

\footnotetext{Source: http://en.wikipedia.org/wiki/ASCII\#ASCII\_printable\_characters}

\begin{equation*}
  \sum_{n=0}^{2^{16}} 94^n
\end{equation*}

I believe this number is colloquially known as a \textit{metric shit-ton}.  Even our circa-1990
16-bit text editor has vastly more states than the number of atoms in the known
universe\footnotemark.  

\footnotetext{Source: http://en.wikipedia.org/wiki/Shannon\_number}

\subsection{EOF}

I can think of two answers to this question.  My first instinct is to answer ``2''; either $right$
is empty, or it is not.  However, this schema includes $Editor$, and according to Mr. Jacky:

\begin{quote}
  \textit{This indicates that all the declarations and predicates in $Editor$ apply to [this schema] as well...}
\end{quote}

This implies that $EOF$ is really an extension of $Editor$, so their state counts are similar.  I
hope to attain enlightenment during our class discussion of this problem.

\end{document}
