\documentclass{article}
\usepackage{bs-zed}

\title{State-Space Tracking Simulator}
\author{Bart Massey et al}
\date{26 February 2009}

\begin{document}
\maketitle

We describe a simple simulator for a robot receiving and reporting signals as it moves up and down a
hallway.

The hallway has a fixed number of locations along it at which the robot can be.

\begin{axdef}
  hallway\_size : \nat
\end{axdef}
\begin{zed}
  POSN == 1 \upto hallway\_size
\end{zed}

The robot receives an arbitrary set of signals.

\begin{zed}
  [SIGNAL, STRENGTH]
\end{zed}

Each of the robot's signals has a known (mapped) strength and variance at each point in the hallway.

\begin{schema}{SignalParameters}
  strength : STRENGTH \also
  variance : \num
\end{schema}
\begin{axdef}
  map : SIGNAL \fun POSN \fun SignalParameters
\end{axdef}

The simulation runs for some amount of time.
\begin{axdef}
  t\_end : \nat
\end{axdef}

At any given time $t$, the robot is at some specific $x$ position.
\begin{schema}{Robot}
  x : POSN \also
  t : \nat
  \where
  t \leq t\_end
\end{schema}

The robot starts at time 0.

\begin{schema}{InitMove}
  Robot
  \where
  t = 0
\end{schema}

For the simulated measurements, we assume a Gaussian-distributed pseudo-random number generator,
with mean and variance according to the actual measurements.  Note that the PRNG is not a function;
it can return any of a number of values for a given signal and variance.

\begin{axdef}
  gaussian\_prng : SignalParameters \rel STRENGTH
\end{axdef}

At each time step, the robot randomly moves left, right, or stays still, measures and reports the
signal strength and position.  The time then steps forward.  The signal strength is obtained from
the gaussian PRNG.

\begin{schema}{Move}
  \Delta Robot \also
  strengths! : SIGNAL \fun STRENGTH \also
  x! : POSN \also
  t! : \nat
  \where
  \forall s : SIGNAL @ strengths!~s \in gaussian\_prng \limg \{ map~s~x \} \rimg \also
  (x' - x) \in ( -1 \upto 1 ) \also
  t' = t + 1 \also
  x! = x \also
  t! = t
\end{schema}

\begin{zed}
  Simulation \defs InitMove \semi Move
\end{zed}

\end{document}
