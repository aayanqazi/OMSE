\documentclass[10pt]{article}
% Way of Z problems 8.2.4.1, 9.3.3.1, 9.3.4.1, 9.4.1.
\usepackage{bs-zed}

\begin{document}

\title{the way of Z \\ Chapters 8 \& 9 Problems}
\author{Ben Straub}
\maketitle


\section{Problem 8.2.4.1}
\begin{quote}
  {\it Is this an erroneous expression?  Why or why not?
    \[ \{red, green\} \union yellow \]}
\end{quote}

% 8.2.4.1 1 min
With Jacky's definitions, this is certainly an error, since $\union$ operates only on sets, and
$yellow$ is not a set.



\section{Problem 9.3.3.1}
\begin{quote}
  {\it Write a fully formal defintiion of the} dept {\it relation that does not use the three dots.}
\end{quote}

% 9.3.3.1 3 min [hint: doing it "the long way" would be silly]
I'm not sure how I would have solved this without reading past chapter 9.  Chapter 10 gives us more
information on combining predicates, which suggests the following form.

\begin{axdef}
  dept \_ : PHONE \rel DEPARTMENT
  \where
  (0000 \leq \_ \leq 0999 \land dept \_ = administration) \lor \\
  (4000 \leq \_ \leq 4999 \land dept \_ = research) \lor       \\
  (6000 \leq \_ \leq 6999 \land dept \_ = manufacturing)
\end{axdef}

We split the problem into three predicates, and combine them to provide an implied set definition.
This isn't really satisfactory, though; we're not defining a set \textit{per se}, we're just
defining a predicate for membership of a set.  With my limited experience, I'm not quite sure
this even qualifies as a function definition; it doesn't seem exact enough.

Chapter 11 gives us another solution: the almighty lambda.

\begin{syntax}
  dept & == & \{ \lambda i : PHONE | 0000 \leq i < 1000 @  administration \} \union \\
       &    & \{ \lambda i : PHONE | 4000 \leq i < 5000 @ research \} \union       \\
       &    & \{ \lambda i : PHONE | 6000 \leq i < 7000 @  manufacturing \}
\end{syntax}

Now we actually define three sets of type $PHONE \cross DEPARTMENT$, and define $dept$ to be their
union.



\section{Problem 9.3.4.1}
\begin{quote}
  {\it Define the} parent {\it relation: Abraham and Sarah are the parents of Isaac, etc.}
\end{quote}

% 9.3.4.1 10 min
Since this isn't a numeric relationship, it seems like the only way to define this relation is
explicitly.

\begin{syntax}
  parent == & \{ & ishmael \mapsto hagar,   \\
            &    & ishmael \mapsto abraham, \\
            &    & isaac \mapsto abraham,   \\
            &    & isaac \mapsto sarah,     \\
            &    & esau \mapsto isaac,      \\
            &    & esau \mapsto rebekah,    \\
            &    & jacob \mapsto isaac,     \\
            &    & jacob \mapsto rebekah \}
\end{syntax}



\section{Problem 9.4.1}
\begin{quote}
  {\it The second elevator leaves the first floor and stops at the third floor while all the other
    elevators remain in place.  This new configuration is named $floor'$.  Define $floor'$ in terms
    of $floor$.}
\end{quote}

% 9.4.1 3 min [see previous hint].
$floor'(x)$ is equal to $floor(x)$ except where $x = 2$.

\[ floor' == floor \oplus \{ 2 \mapsto 3 \} \]

\end{document}
