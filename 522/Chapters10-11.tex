\documentclass[10pt]{article}
% Way of Z exercises 10.4.4.2, 10.4.5.1, 10.6.1, 10.8.1, 11.1.2, 11.3.1, 11.5.1
% Due Feb. 3
\usepackage{bs-zed}

\begin{document}
\title{the way of Z \\ Chapters 10 \& 11 Problems}
\author{Ben Straub}
\maketitle


\section{Exercise 10.4.4.2}
% 10.4.4.2 15min
\begin{quote}
  {\it Consider expressing case analyses using equivalence.  Compare this definition of temperature
    and phase to the previous one.  Which logical connective is used to separate cases?  What happens
    at $temp = 0$?  At $temp = 100$?
    \begin{axdef}
      temp : TEMP \\
      phase : PHASE
      \where
      temp \leq 0 \iff phase = solid \\
      0 \leq temp < 100 \iff phase = liquid \\
      temp > 100 \iff phase = gas
    \end{axdef}
  }
\end{quote}

There are two areas of concern with this definition.  The first concerns the use of equivalence in
the definition, which appears to be correct, though perhaps inappropriate.  Take the first line:
\[ temp \leq 0 \iff phase = solid \]
which expands to:
\[ (temp \leq 0 \implies phase = solid) \land (phase = solid \implies temp \leq 0) \land \also
   (temp > 0 \implies phase \neq solid) \land (phase \neq solid \implies temp > 0)\]
This is correct, but since the other statements cover the cases where $temp > 0$, the additional
restrictions provided by $\iff$ aren't really necessary.  This is like making a garden shed out of
foot-thick steel-reinforced concrete.

The second area of concern is the coverage of $temp$.  It appears that two of the statements apply
at $temp = 0$ ($phase = solid \land phase = liquid$, which is mathematically impossible), and
none apply when $temp = 100$.

\section{Exercise 10.4.5.1}
% 10.4.5.1 15min
\begin{quote}
  {\it Write the table of situations that satisfies each of these predicates.  Would these make
    reasonable saety requirements?\footnote{The solution to this exercise is given in Appendix F of
      the book.} }
    \begin{eqnarray}
      beam = on  & \iff  & door = closed \\
      beam = on  & \land & door = closed \\
      beam = on  & \lor  & door = open   \\
      beam = off & \lor  & door = closed
    \end{eqnarray}
\end{quote}
\begin{enumerate}
\item
  \begin{tabular}{l||l|l}
    $beam$ & $on$     & $off$ \\
    \hline
    $door$ & $closed$ & $open$
  \end{tabular}

  This implies that the beam turns on as soon as the door is closed, which is not a particularly
  good safety requirement.

\item
  \begin{tabular}{l||l}
    $beam$ & $on$ \\
    \hline
    $door$ & $closed$
  \end{tabular}

  This predicate says nothing about when the door is open, nor when the beam is off, so it's not
  complete enough to be a safety requirement.

\item
  \begin{tabular}{l||l|l|l}
    $beam$ & \textcolor{red}{$on$}   & $on$     & $off$ \\
    \hline
    $door$ & \textcolor{red}{$open$} & $closed$ & $open$
  \end{tabular}

  The situation in red worries me.

\item
  \begin{tabular}{l||l|l|l}
    $beam$ & $off$  & $off$    & $on$ \\
    \hline
    $door$ & $open$ & $closed$ & $closed$
  \end{tabular}

  This is actually equivalent to $beam=on \implies door=closed$, which is our original safety
  requirement, and which I would argue is better at conveying intent.
\end{enumerate}



\section{Exercise 10.6.1}
% 10.6.1 10min
\begin{quote}
  {\it Define the relation $phase_R$ from temperature to phase, consistent with the definitions of
    $temp$ and $phase$ in section 10.4.2.  }
\end{quote}

This is really just reorganizing the predicates in the original axiomatic definition so that they
define a set.  I'll follow the union-of-lambdas pattern I used in the last homework:

\begin{syntax}
  phase_R ==                                                         \\
  \t2 \{ \lambda t : TEMP | t < 0             & @ & solid \} \union  \\
  \t2 \{ \lambda t : TEMP | 0 \leq t \leq 100 & @ & liquid \} \union \\
  \t2 \{ \lambda t : TEMP | t > 100           & @ & gas \}
\end{syntax}



\section{Exercise 10.8.1}
% 10.8.1 5min
\begin{quote}
  {\it What is the meaning of this predicate when $f\ x$ is undefined?  When $f\ x$ is
    defined?\footnote{The solution to this exercise is given in Appendix F of the book.} }
  \[ x \in \dom f \implies f\ x = y \]
\end{quote}

The $\implies$ relation holds either when both sides are true, or the left side is false.  So when
$x$ is in the domain of $f$ (``$f\ x$ is defined''), the truth of this statement is equal to the
truth of $f\ x = y$.

The other case is when $x \notin \dom f$ (``$f\ x$ is not defined''), in which case the statement is
true.  Lovely that $\implies$ lets us bypass the whole function-not-defined problem.



\section{Exercise 11.1.2}
% 11.1.2 20min
\begin{quote}
  {\it Define the set of points within the rectangular $window$ whose lower left and upper right
    corners are points modelled by pairs of integers $(x_l, y_l)$ and $(x_u, y_u)$, respectively.
    Then define $segment$, the portion of the previously defined $line$ that lies within $window$.
  }
\end{quote}

For $window$, we comprehend the set of points within that window; that is, the points whose $x$
coordinates are between $x_l$ and $x_u$, and likewise for $y$.

\begin{syntax}
  window == \{ x,y : \num | (x_l \leq x \leq x_u) \land (y_l \leq y \leq y_u) \}
\end{syntax}
To define $segment$, we simply find all the points common to both sets.

\begin{syntax}
  segment == line \cap window
\end{syntax}



\section{Exercise 11.3.1}
% 11.3.1 5min
\begin{quote}
  {\it Is our second definition of $PRIME$ a constructive definition?  Could this definition be
    translated to an executable program in some suitable language? 
  \[ \nat_2 == \nat \setminus \{ 0,1 \} \\
  PRIME == \nat_2 \setminus \{n,m : \nat_2 @ n*m\}\] }
\end{quote}

Jacky says a constructive definition is one that has ``the quantity of interest [...] by itself on
one side of the equation, and the other side only uses operators we have available in our
programming language [...]''  The only hangup we have here is the use of infinite sets, but these
can be represented at runtime by a set that is always finite, but grows with time.  The Sieve
program I wrote (a long time ago) worked like this; it could only deal with a subset of natural
numbers at a time, but it kept a list of the primes it found, and it could deal with arbitrarily
large integers.  Given infinite time, it could generate all the primes, so my answer is yes.



\section{Exercise 11.5.1}
% 11.5.1 5-45min 
\begin{quote}
  {\it Define the function $month$ that extracts the $MONTH$ component from a $DATE$.  }
\end{quote}

This is a simple pattern match; for any given $(d,m,y)$ tuple, we just want $m$.

\begin{axdef}
  month : DATE \fun MONTH 
  \where
  \forall d : DAY, m : MONTH, y: YEAR @ (d,m,y) \mapsto m
\end{axdef}


\end{document}