\documentclass[11pt]{article}

\usepackage[pdfborder=false]{hyperref} % Remove borders from PDF links
\usepackage{amsmath} % for equation* environment
\usepackage{oz} % for zed macros
\usepackage{times} % for nice-looking text
\setlength{\parindent}{0pt} % for less indentation
\setlength{\parskip}{1em} % for less whitespace
\usepackage{fullpage} % for smaller margins


\begin{document}

\title{Strip Poker in Z}
\author{Mike Bagdanoff \& Ben Straub}
\maketitle

In the early days of man, fruits and grains were fermented to produce an beverage that produces
euphoria and a lack of inhibitions in its consumer.  This gave rise to the drinking game, and
eventually (when sex was invented) strip poker became a favorite drinking game of those who like
looking at other naked people.  Througout its history, the game has been plagued by a simple, yet
insidious flaw; once a group of people are drunk enough to play, they are usually too drunk to agree
on rules that level the playing field and produce the outcome desired by the males in the group.
This is why we have decided to encode a simple set of rules for the reward\/penalty portion of the
game.  These rules can be applied to any strip game, including Strip Blackjack, Strip Arm-Wrestling,
and Strip Yelling.

First off, we need to define valid clothing types.  The general rule is that if you buy it in a
pair, it comes off in a pair, hence the plural forms of $shoe$ and $sock$.  Also there is only a
notion of $top$ and $bottoms$; regardless of how many layers you are wearing, all the top pieces of
the outfit come off at once, and skirt, pants, whatever -- it all comes off together.  Since guys
don't wear bras, underwear is one piece of clothes; bras come off along with panties.

A $HANDRANK$ is simply a ranking of the value of the relative hands.  The assignment of a handrank
is done outside the scope of this definition; we just care who has the highest ranking hand, or is
the loudest, depending on the game.

Finally we declare what the penalties may be after each round: take a drink, dance around, or
\textit{\tiny (ahem)} service the winner (if the game has progressed that far).

\begin{zed}
  ARTICLE ::= top | bottoms | socks | shoes | underwear \also
  CLOTHES == \power ARTICLE \also
  HANDRANK == \nat \also
  PENALTY ::= drink | dance | servicewinner | none
\end{zed}

We want to know who has the highest HANDRANK and since in some versions of Poker or other games
there can be a tie, we need to allow for multiple winners to be identified.
\begin{axdef}
  Winners : \seq HANDRANK \rel \{ \nat \}
  \where
  \forall r : \seq HANDRANK @ \\
  \t1 Winners~r = \dom (r \rres \{max (\ran r) \})
\end{axdef}

A player comes into the game with an original set of clothes that he or she never actually parts
with, regardless of the clothes that they are currently wearing.  Removed clothes are placed in a
pile by the player and saved for later.  A player also can be penalized in certain situations.
\begin{schema}{Player}
  wearing : CLOTHES \\
  original : CLOTHES \\
  penalty : PENALTY
\end{schema}

This game can only be played with 2 or more players and once the game has started, no new players
can be added as that would just be lame.  If you couldn't get the nerve to play when everyone else
started, go clean up some of the bottles and cans lying around or just watch.  Jerk.
\begin{axdef}
  NumPlayers : \nat
  \where
  NumPlayers \geq 2
\end{axdef}

\begin{schema}{Game}
  players : \seq Player
  \where
  1 < \# players <= NumPlayers
\end{schema}

At the beginning of the game everyone is wearing the standard set of articles.  If a player is not
wearing socks, they can pretend they are, or borrow socks from someone wearing two pairs.
\begin{schema}{GameInit}
  Game
  \where
  \forall p : \ran players @ \\
  \t1 p.wearing = p.original = \{top, bottoms, socks, shoes, underwear\}
\end{schema}

Now we get to the meat of the game.  When a player is commanded to strip, they must remove one piece
of clothing and take a drink.  If they are already completely nude, then they must still take a
drink and then choose either to dance about in the manner of their choosing, or (if the party has
progressed to a certain point) do something awesome to/for the winner(s) of the hand.
\begin{axdef}
  Strip : Player \rel Player
  \where
  \forall p,p' : Player | p' = Strip~p @ \\
  \t1 p'.original = p.original \land \\
  \t1 p'.wearing \subseteq p.wearing \land \\
  \t1 ((\#(p.wearing) > 0 \land \#(p'.wearing) = \#(p.wearing)-1 \land \\
  \t2 p'.penalty = drink) \lor \\
  \t1  (p'.wearing = p.wearing = \{ \} \land \\
  \t2 (p'.penalty \in \{ drink, dance\} \lor \\
  \t2 p'.penalty \in \{ drink, servicewinner \})))
\end{axdef}

A winning player gets to put a piece of clothing back on, as long as it is a piece of their own
clothing.  Switching clothes around is a different game, and usually leads to stretched out dresses.
\begin{axdef}
  Dress : Player \rel Player
  \where
  \forall p,p' : Player | p' = Dress~p @ \\
  \t1 p'.penalty = none \land \\
  \t1 p'.original = p.original \land \\
  \t1 p'.wearing \subseteq p.original \land \\
  \t1 ((p.wearing = p.original \land Dress~p = p) \lor \\
  \t2  (\#(p'.wearing) = \#(p.wearing)+1))
\end{axdef}

Once the hand has been dealt and the game has been played, someone calls, and the hands are ranked.
There is a set of winners who get to put a piece of clothing on, and a bunch of people who now have
to take a piece of clothing off.  Poor dears.
\begin{schema}{Call}
  \Delta Game \\
  hands?  : \seq HANDRANK
  \where
  \forall i : \dom players; p,p' : Player | p = players~i; p' = players'~i @ \\
  \t1 (i \notin Winners~hands?  \land p' = Strip~p) \lor \\
  \t1 (i \in Winners~hands?  \land p' = Dress~p)
\end{schema}

It all boils down to making sure that everyone has the right amount and type of clothing on, and
then playing each hand until people start getting bored of being naked, pass out, or \ldots
whatever.
\begin{zed}
  T\_Game == GameInit \semi Call
\end{zed}

\end{document}