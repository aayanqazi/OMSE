\documentclass[10pt]{article}
% Syllabus: 12.3.1, 12.3.2, 12.4.1.1, 13.2.1, 15.1.1
% Assignment: 12.3.2, 12.4.1.1, 13.2.1, 15.1.1

% Note that 12.3.1 and 12.3.2 are the exercises at the end of section 12.3 that are not labeled in
% the text.

% Due Feb. 10
\usepackage{bs-zed}

\begin{document}
\title{the way of Z \\ Chapters 12, 13 \& 15 Problems}
\author{Ben Straub}
\maketitle

\section{Exercise 12.3.1}
\begin{quote}
  {\it
    Expand $\forall Editor | Init @ EOF$.
  }
\end{quote}
Read aloud, this says that for all Editor states where Init holds, EOF holds as well. To verify
this, we just need to expand out $Init \land EOF$ in the $Editor$ context and see if there are any
contradictions.

For reference:
\begin{schema}{Editor}
  left, right : TEXT
  \where
  \#(left \cat right) \leq maxsize
\end{schema}
\begin{schema}{Init}
  Editor
  \where
  left = right = \langle \rangle
\end{schema}
\begin{schema}{EOF}
  Editor
  \where
  right = \langle \rangle  
\end{schema}

Given these, the expression expands to:

\begin{syntax}
  \forall left,right : TEXT | (\#(left \cat right) \leq maxsize) \land (left = right = \langle \rangle) @ \\
  \t1 right = \langle \rangle
\end{syntax}
\tocheck



\section{Exercise 12.3.2}
\begin{quote}
  {\it
    What does $S\_Division$ say about $status$ when the $DivideByZero$ condition does not occur?
  }
\end{quote}

The expression $a \iff b$ implies $(a \land b) \lor (\lnot a \land \lnot b)$.  So when $S\_Division$
says that $DivideByZero \iff status = error$, this means 
\[ (DivideByZero \land status = error) \lor
   (\lnot DivideByZero \land status \neq error) \]
which, since $STATUS$ has only two values, really says that $status = ok$.
\tocheck


\section{Exercise 12.4.1.1}
\begin{quote}
  {\it
   What is the precondition of $ForwardTwo$? 
  }
\end{quote}

We could expand out $ForwardTwo$ explicitly using primes and double primes:

\begin{schema}{ForwardTwo}
  left, right, left', right', left'', right'' : TEXT \\
  ch?, ch?' : CHAR
  \where
  \#(left \cat right) \leq maxsize\\
  \#(left' \cat right') \leq maxsize\\
  \#(left'' \cat right'') \leq maxsize\\
  ch? = ch?' = right\_arrow \\
  right \neq \langle \rangle \\
  \textcolor{blue}{right' \neq \langle \rangle} \\
  left' = left \cat \langle head(right)\rangle \\
  right' = tail(right) \\
  left'' = left' \cat \langle head(right')\rangle \\
  right'' = tail(right') \\
\end{schema}

The most restrictive precondition is highlighted in blue: $right' \neq \langle \rangle$.  This means
that $\#right \geq 2$ or our editor will crash.

As a side note, the $ch? = ch?' = right\_arrow$ restriction is a bit confusing.  It almost seems as
if Z is encouraging us to separate input classification (recognizing which key was pressed) from
actual state transitions (moving the cursor); combining the two gives you this wierd predicate where
a transition with no middle state requires that the same key be pressed twice.

\tocheck



\section{Exercise 13.2.1}
\begin{quote}
  {\it
    Define the operation $Next$ that advances a $Date$ by one day.
  }
\end{quote}
\tostart



\section{Exercise 15.1.1}
\begin{quote}
  {\it
    Equality $=$, greater than $>$, and less than $<$ are all transitive. Is inequality $\neq$
    transitive as well?
  }
\end{quote}
\tostart



\end{document}