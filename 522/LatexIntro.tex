\documentclass[11pt]{article}

\usepackage[pdfborder=false]{hyperref} % Remove borders from PDF links
\usepackage{zed-csp}                   % for zed macros
\usepackage{times}                     % for nice-looking text
\usepackage[usenames]{color}           % for color names
\usepackage{fullpage}                  % for smaller margins
\setlength{\parindent}{0pt}            % for less indentation

\begin{document}

\title{A Quick Intro to \LaTeX}
\author{Ben Straub}
\maketitle

\section{WTF?}

It's a lot like writing source code for your documents.

\section{Toolchain}

In order to use this system, you'll need a toolchain, or \textit{distribution}.  There are lots of
these.

\subsection{Windows}
I'll recommend MiKTeX on Windows, for a reason we'll see later.  To get MiKTeX, go to the
{\color{blue} \href{http://miktex.org/2.7/setup}{MiKTeX download page}}.  Download the 80MB ``Basic
Installer'' and run it.

\subsection{OS X}
I use {\color{blue} \href{http://www.tug.org/mactex/}{MacTeX}} on OS X.


\section{First Steps}
Now you're ready to typeset something.  Here's some Z:

\begin{schema}{Ignore}
  \Xi Editor \\
  ch ? : CHAR
  \where
  ch? \notin printing \union {backspace, forward, delete ...}
\end{schema}

Copy the {\color{blue} \href{http://www.straubnet.net/LatexIntro.tex}{source of this document}} into
a text editor, and save it as \texttt{test.tex}. Now open up a command prompt, and type
\texttt{pdftex test.tex}.

MiKTeX doesn't come with all the ``packages'' (read: third-party libraries) this document needs, so
it'll have to stop in the middle and download them.  It also can't install more than one package at
a time, so you'll probably have to run it several times. When you see a line like \texttt{LaTeX
  finished at Fri Jan 23 06:34:13}, you know you're done.


\section{Making it Easy}
Copy this into a file called \texttt{my-zed.sty}, and keep it with your homework for this class:

\begin{verbatim}
\usepackage[pdfborder=false]{hyperref} % Remove borders from PDF links
\usepackage{zed-csp}                   % for zed macros
\usepackage{times}                     % for nice-looking text
\usepackage[usenames]{color}           % for color names
\usepackage{fullpage}                  % for smaller margins
\setlength{\parindent}{0pt}            % for less indentation
\end{verbatim}

Then setting up for a new homework becomes easy:

\begin{verbatim}
\documentclass[11pt]{article}
\usepackage{my-zed}
\begin{document}

% ... Content goes here ...

\end{document}
\end{verbatim}

\section{\LaTeX and Z}
The \texttt{zed-csp} package we're using provides commands for all the Z symbols, but it's not
always obvious what they are.  There's a handy reference at {\color{blue}
  \href{http://www.ctan.org/tex-archive/macros/latex/contrib/zed-csp/}{CTAN}} (look for
\texttt{zed2e.pdf}), though you should ignore the first two sections, they're pretty outdated.

\end{document}