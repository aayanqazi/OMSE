\documentclass{article}
\usepackage{bs-zed}
\usepackage{fullpage}

\title{Vanity Plates in Z}
\author{Bart Massey et al}
\date{26 February 2008}

\begin{document}
\maketitle

Some prose goes here.  A plate consists of a sequence of characters, etc. etc. etc.

\begin{zed}
  [ALPHA, DIGIT] \also
  CHAR ::= alpha\ldata ALPHA \rdata | digit\ldata DIGIT \rdata | hyphen | half\_hyphen | space |
  half\_space \also
  ALPHAS == \ran alpha \also
  DIGITS == \ran digit
\end{zed}

A candidate is a sequence of characters that passes a couple of syntactic checks.  All plates have
between 1 and 7 characters, and at most 6 of them can be alphanumeric.  At most one can be
punctuation.

\begin{axdef}
  syntactic\_plates : \power (\seq CHAR)
  \where
  \forall p : syntactic\_plates @ \\
  \t1 1 \leq \#p \leq 7 \land \\
  \t1 1 \leq \#(p \rres (ALPHAS \cup DIGITS)) \leq 6 \land \\
  \t1 0 \leq \#(p \rres \{ half\_space, half\_hyphen \}) \leq 1
\end{axdef}

Plates with three letters followed by three numbers, or vice versa, are also illegal, presumably
because they are deceptive.  Note that plates with punctuation (spaces, hyphens, etc) are implicitly
not deceptive by this definition.

\begin{zed}
  three\_numbers == \{ tn : \seq DIGITS | \#tn = 3 \} \also
  three\_letters == \{ tl : \seq ALPHAS | \#tl = 3 \} \also
  letter\_number\_plates == \\
  \t1 \{ n : three\_numbers; l : three\_letters @ n \cat l \}
  \also
  number\_letter\_plates == \\
  \t1 \{ n : three\_numbers; l : three\_letters @ l \cat n \}
  \also
  deceptive\_plates == number\_letter\_plates \cup letter\_number\_plates
\end{zed}
p
A candidate plate is a syntactically valid plate that is not deceptive.

\begin{zed}
  candidate\_plates == syntactic\_plates \setminus deceptive\_plates
\end{zed}

\end{document}
